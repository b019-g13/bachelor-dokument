\cleardoublepage
\chapter{Implementasjon / Produksjon / Gjennomføring (Generisk tittel)}
\label{chap:implementation} 
\clearpage

% \meta{
% Her skal det beskrives hvordan man faktisk produserte resultatene i prosjektet, og viktigst, beskrive selve produktet. Hvilke verktøy brukte man, hvordan foregikk produksjonen, etc. Utformingen av dette kapittelet avhenger helt klart av type prosjekt.
% }

I dette kapittelet skal vi presentere resultatene i prosjektet og en beskrivelse av nettstedet. Her vil vi også presentere hvilke verktøy vi endte opp med å bruke og hvordan selve produksjonen foregikk.

\section{Verktøy}
Vi endte opp med å bruke flere verktøy enn beskrevet i  \ref{sec:technologies}. 
I tillegg til hovedverktøyene ble følgende verktøy og teknologier benyttet:

\subsection{GNU/Linux}
GNU/Linux er en familie med Unix-lignende operativsystemer som baserer seg på Linux-kjernen \cite{kernel_org} og en del programvare fra GNU-prosjektet \cite{gnu_org}. I denne familien finner vi blant annet Ubuntu, Debian, Fedora, Red Hat Linux og Arch Linux.

I dette prosjektet blir GNU/Linux benyttet som operativsystem på serveren som \q{hoster} back-end.

\subsection{MySQL}
MySQL \cite{oracle2019am} er et databasestyringssystem med åpen kildekode.

Prosjektets database er en MySQL-database.

\subsection{Let’s Encrypt}
Let’s Encrypt \cite{le2019ale} er en gratis og åpen sertifikatautoritet som gir ut X.509-sertifikater for kryptering av transportlaget. Tjenesten leveres av ISRG (Internet Security Research Group).

Studentgruppen har benyttet Let's Encrypt til å lage TLS-sertfikater\footnote{Se avsnitt \ref{sec:analysis-security-tls}}.

\subsection{Node.js}
Node.js er et servermiljø med åpen og gratis kildekode. Node.js kan kjøre på ulike plattformer (Windows, Linux, Unix, Mac OS X) og bruker JavaScript på serveren \cite{w3schools2019win}. I tillegg kan Node.js generere dynamisk sideinnhold og opprette, åpne, lese, skrive, slette og lukke filer på serveren. I tillegg kan Node.js legge til, slette og endre data i databasen.

Vi har brukt Node.js for å kunne kjøre verktøy knyttet til utvikling av front-end.

\section{Tilhørende teknologier og begreper}
Ved å bruke verktøyene som har blitt presentert benyttes også noen tilhørende teknologier. Disse blir presentert videre i dette kapittelet. 

\subsection{Git}
Git \cite{TechTarget} er et system for versjonskontroll. Et versjonskontrollsystem blir hovedsaklig brukt under utvikling av software og nettsteder. Det kan også brukes til andre type prosjekter som grafisk design og skriving av dokumenter.

Ved å bruke Git opprettes det historikk over alle endringer, samt en sikkerhetskopi av alle versjoner av filene.

En annen fordel ved å bruke Git er at det blir enklere å samarbeide med andre, uten å måtte tenke på at alle må sitte på nyeste versjon av filene.

\subsection{Github}
GitHub.com er et nettsted for å hoste git repositories, og blir mye brukt for \q{Open Source}-prosjekter. 

Github \cite{TechTarget} er et sentralisert punkt for å samle en brukers repositories. Ved siden av å hoste repositories, lar GitHub brukere dele repositories med hverandre, lage informasjonssider om prosjektet og opprette saker (issues).

\subsection{HTTP/2}
HTTP/2 \cite{Belshe2015httpv} er en revisjon av HTTP-nettverksprotokollen. HTTP er et sett med regler for overføring av filer (tekst, grafiske bilder, lyd, video og andre mediefiler). Et av de store målene med HTTP/2 var å tillate multipleksing.

\subsection{REST-API}
Et API (programmeringsgrensesnitt) er et sett med funksjoner, prosedyrer, metoder eller klasser som brukes av dataprogrammer for å be om tjenester fra operativsystemet eller programvare som er på datamaskinen. En programmerer kan bruke API-er til å lage applikasjoner.

REST (Representational State Transfer) er en arkitektonisk stil for programvare. Et REST-API \cite{Masse2011radr} er et API som følger REST-stilen og de begrensninger som REST definerer.

\subsection{Google Analytics}
\label{sec:google-analytics}
Google Analytics \cite{google2019gtk} er en gratis, webbasert tjeneste som gir statistikk og grunnleggende verktøy for analyse av bruksdata, søkemotoroptimalisering og markedsføring.

\section{Utviklingen}

Det ble opprettet en \q{Branch protection rule} på back-end repository i Github. Dette gjør at det må kjøres en \q{Code review} før kode kan sendes inn til \q{Master branch}-en. Alle medlemmene på studentgruppen har en egen branch.
Når koden skal commites må det opprettes en pull request, som må godkjennes. Da blir det mulig å luke ut kode som inneholder feil som potensielt kan ødelegge for eksisterende kode.

\section{Back-end}
Back-end har blitt utviklet av Bereket og Bjørnar.

%--- Bjørnar - start ---

\subsection{Installasjon og oppsett av Laravel og tilhørende verktøy}
Første steg var å installere package manageren Composer. Deretter kunne Laravel settes opp og installeres. Fordi vi benyttet oss av Composer ble det mulig å installere Laravel med et par kommandoer i terminalen:
\begin{lstlisting}[caption={Installasjon av Laravel med Composer},language=bash]
    composer global require laravel/installer
    laravel new sirkus-media-back-end
\end{lstlisting}
Det var siste versjon av rammeverket som ble lastet ned (5.7).

Til slutt måtte vi konfigurere rammeverket slik at det kunne håndtere databasetilkobling, sending av mail og lagring av opplastede og genererte filer. Laravel samler sine innstillinger i PHP-filer i \q{config} mappen, samt en .env fil som holder på passord og annen sensitiv informasjon vi ikke vil ha i git.

\subsubsection{Database}
Vi begynte med å sette opp tilkobling til databasen. Dette ble gjort ved å først logge inn i MySQL og opprette en bruker og en database. Informasjon for dette ble skrevet inn i .env filen. Se figur under for eksempel:
\begin{lstlisting}[caption={Laravel .env database secrets},language=bash]
    DB_CONNECTION=mysql
    DB_HOST=127.0.0.1
    DB_PORT=3306
    DB_DATABASE=sirkusmedia
    DB_USERNAME=bruker
    DB_PASSWORD=passord
\end{lstlisting}

\subsubsection{Mail}
Etter at databasen var klar for bruk, begynte vi å sette innstillingene for e-post. Vi bestemte oss for å bruke Mailtrap\footnote{\url{https://mailtrap.io/}} under utvikling av løsningen. Mailtrap\cite{mailtrap2019setfsad} er en uekte SMTP-server som kan brukes for testing av e-post. Etter å ha registrert bruker hos Mailtrap fylte vi ut følgende informasjon i .env filen:
\begin{lstlisting}[caption={Laravel .env mail secrets}, language=bash]
    MAIL_DRIVER=smtp
    MAIL_HOST=smtp.mailtrap.io
    MAIL_PORT=2525
    MAIL_USERNAME=bruker
    MAIL_PASSWORD=passord
    MAIL_ENCRYPTION=tls
\end{lstlisting}

Når løsningen skal ut i produksjon må det brukes noe annet enn Mailtrap. Sirkus Media kan selv bestemme hva de ønsker å bruke, men systemet er satt opp til bruk av Mailgun\footnote{\url{https://www.mailgun.com/}}.

\subsubsection{Lagring}
Lagring av opplastede og genererte filer foregår på samme server som back-end kjører på. For å sette opp dette kjørte vi en Larvel Artisan kommando:
\begin{lstlisting}[caption={Laravel Artisan kommando for å sette opp lagring av filer}, language=bash]
    php artisan storage:link
\end{lstlisting}
Storage klassen til Larvel gjør at Sirkus Media enkelt kan benytte seg av andre løsninger, som for eksempel Amazon Web Servies S3.

\subsubsection{Laravel Mix}
Laravel Mix\footnote{\url{https://laravel-mix.com/}} var det siste som måtte settes opp før utviklingen kunne starte. Dette er en \textit{wrapper} rundt den populære module bundleren Webpack\footnote{\url{https://webpack.js.org/}}. For å bruke Laravel Mix måtte vi først installere Node.js. Etter Node.js ble installert trengte vi kun å kjøre et par npm-kommandoer:
\begin{lstlisting}[caption={npm-kommando for å sette opp front-end til Laravel}, language=bash]
    npm install
    npm run dev
\end{lstlisting}

For å konfigurere hvilke filer som skal brukes av front-end til Laravel fylte vi ut filen webpack.mix.js:
\begin{lstlisting}[caption={Eksempel på oppsett av webpack.mix.js}]
    const mix = require("laravel-mix");

    mix.sass("resources/sass/app.scss", "public/css").version();
    mix.js("resources/js/app.js", "public/js").version();
    mix.copy("resources/images", "public/images").version();

    mix.browserSync({ proxy: "localhost:8000", notify: false });
\end{lstlisting}

Når det nevnes \q{front-end til Laravel} så er det ikke snakk om front-end til nettsiden som lages for Sirkus Media, men front-enden til CMS-et.
Laravel kommer med en del dependencies til front-end som vi ikke trenger. Alle dependencies defineres i package.json. Følgende dependencies ble fjernet: Bootstrap, jQuery, lodash, Popper, Vue. Det ble kun lagt til en dependency; Shopify sin Draggable pakke. BrowserSync ble også lagt til som en devDependency.

\begin{lstlisting}[caption={Orginal package.json}]
    "devDependencies": {
        "axios": "^0.18",
        "bootstrap": "^4.0.0",
        "cross-env": "^5.1",
        "jquery": "^3.2",
        "laravel-mix": "^4.0.7",
        "lodash": "^4.17.5",
        "popper.js": "^1.12",
        "resolve-url-loader": "^2.3.1",
        "sass": "^1.15.2",
        "sass-loader": "^7.1.0",
        "vue": "^2.5.17"
    },
    "dependencies": {}
\end{lstlisting}

\begin{lstlisting}[caption={Siste package.json}]
    "devDependencies": {
        "browser-sync": "^2.26.3",
        "browser-sync-webpack-plugin": "^2.0.1",
        "cross-env": "^5.1",
        "laravel-mix": "^4.0.7",
        "resolve-url-loader": "^2.3.1",
        "sass": "^1.15.2",
        "sass-loader": "^7.1.0",
        "vue-template-compiler": "^2.6.10"
    },
    "dependencies": {
        "@shopify/draggable": "^1.0.0-beta.8",
        "axios": "^0.18.0"
    }
\end{lstlisting}
Merk at vue-template-compiler ligger som en dependency her kun fordi siste versjon av Laravel Mix ikke kjører uten.

\subsection{Modeller og database migrations}
Definering av modeller og migrasjon filer til databasetabellene var det første som ble gjort etter konfigurering av rammeverket.
Modeller lagde vi ved å skrive følgende Larvel Artisan kommando i terminalen:
\begin{lstlisting}[caption={Laravel Artisan kommando for oppretting av modell og migration},language=php]
    php artisan make:model Modelnavn -m
\end{lstlisting}
Parameteret \q{-m} i kommandoen over brukes for å opprette en migration fil for modellen. Det er også mulig å opprette migration fil etter å ha lagd modell:
\begin{lstlisting}[caption={Laravel Artisan kommando for oppretting av migration fil},language=php]
    php artisan make:migration migration_file_name
\end{lstlisting}
Etter at modellene og migration filene er opprettet kan man definere forhold mellom modellene (gjøres i modell filen), samt databasetabell for hver modell (gjøres i migration filen).
Når migration filene er ferdig definert kan man kjøre migration ved å skrive følgende kommando i terminalen:
\begin{lstlisting}[caption={Laravel Artisan kommando for å kjøre migration},language=php]
    php artisan migrate
\end{lstlisting}
Etter at migration blir kjørt vil alle tabellene og deres kolonner bli opprettet i MySQL.
Under er et eksempel på en migration fil
\lstinputlisting[caption={Laravel migration fil}, language=php]{laravel-code-examples/migration.php}

\subsubsection{Eloquent forhold mellom modellene}
Eloquent\footnote{\url{https://laravel.com/docs/5.8/eloquent}} er Laravel sitt ORM, det lar oss definere forhold mellom de forskjellige modellene. Det er ønskelig å definere forhold som f.eks: en forfatter har mange bøker. Når et slikt forhold er definert i modellene til Larvel lar Eloquent oss hente alle bøkene til forfatteren med kode som dette: \lstinline[language=PHP]{$author->books}.

For å definere slike forhold brukte vi de innebygd metodene til Laravel som: belongsToMany, hasOne, belongsTo og hasMany. For eksempel av bruk, se følgende kode:
\lstinputlisting[caption={Laravel modell med et forhold}, language=php]{laravel-code-examples/model-relationship.php}

\subsubsection{Testing av migration filer}
Første gang vi kjørte migration dukket det opp en feil:
\begin{lstlisting}[caption={Feilmelding i Laravel ved kjøring av migration},language=PHP]
Illuminate\Database\QueryException : SQLSTATE[42000]: Syntax error or access violation: 1071 Specified key was too long; max key length is 767 bytes (SQL: alter table `users` add unique `users_email_unique`(`email`))
\end{lstlisting}
Denne feilen viste seg å være relativt vanlig og kom av at vi kjørte en eldre versjon (5.7) av MySQL som ikke støtter datatypen varchar med lengde på 255 når databasen sin collation er satt til UTF8MB4. Løsningen her ble å sette standard varchar lengde til 191.
\begin{lstlisting}[language=PHP, caption={Definering av standard varchar lengde i AppServiceProvider.php}]
    <?php

    class AppServiceProvider extends ServiceProvider
    {
        public function boot()
        {
            Schema::defaultStringLength(191);
        }
    }
\end{lstlisting}

\subsubsection{Testing av forhold mellom modellene}
Når alle forhold var ferdig definert og migration filene kjørte uten feil, fylte vi databasen med data for å teste at alle forhold var riktig satt opp. En interessant feil vi støtte på er relatert til vår bruk av UUID som primary key i stedet for en inkrementerende integer i flere av modellene. Når vi forsøkte å lagre en modell til databasen fikk vi denne feilen:
\begin{lstlisting}[language=PHP]
    SQLSTATE[HY000]: General error: 1364 Field 'id' doesn't have a default value ...
\end{lstlisting}

Feilen kom av hvordan Laravel har satt opp standardmodell alle modeller baserer seg på.
Vi måtte finne en måte å si at alle modeller som brukte UUID har primary key av typen string, ikke skal inkrementeres og samtidig automatisk generere en UUID ved lagring. Den beste måten vi fant å gjøre dette på var ved å lage en trait. Følgende fil ble opprettet:
\lstinputlisting[caption={Trait for å håndtere UUID for Larvel-modeller}, language=php]{laravel-code-examples/trait-uuid.php}

Når trait-en var definert, kunne vi bare inkludere den i alle modeller som brukte UUID som primary key, som vist i figur under, så var problemet løst.
\begin{lstlisting}[caption={Bruk av UUID trait i modell}, language=PHP]
    <?php

    use App\Traits\UsesUuid;

    class User ...
    {
        use UsesUuid;

        ...
    }
\end{lstlisting}

Etter de overnevnte problemene ble løst, støtte vi ikke på noen flere nevneverdige problemer rundt forhold mellom modeller.

\subsection{Oppsett av autentisering}
Med databasen og forhold mellom modeller på plass var vi klare for å begynne utviklingen av selve CMS-et, men vi bestemte oss for å først sette opp autentisering da dette er enklere å implementere fra starten av, fremfor å gjøre det til slutt.
Laravel gir oss veldig mye gratis på dette området, ved å kjøre en kommando settes opp alt vi trenger:
\begin{lstlisting}[language=PHP]
    php artisan make:auth
\end{lstlisting}
Etter å ha kjørt kommandoen over får vi nødvendige controllere og views for login og registrering av brukere, samt system for tilbakestilling av passord.




% Autentisering brukes til å kontrollere tilgang til ressurser. For å kontrollere tilgang skal nettsiden ha logginn og registreringsside. Registrering siden skal være tilgjengelig for brukere som logget seg inn.
% Lagte autenisering ved å kjøre koden i comandlijen.
% \begin{lstlisting}[language=PHP]
%     php artisan make:auth
% \end{lstlisting}

% I RegisterController kan det endres tilgangen for å opprette nye brukere.
% \begin{lstlisting}[language=PHP]
%     $this->middleware('guest');
% \end{lstlisting}
% Endre guest med auth. Dette krever at bruker må logge seg inn for å kunne opprette nye brukere.
% \begin{lstlisting}[language=PHP]
%     $this->middleware('auth');
% \end{lstlisting}

% \subsubsection{ Registrering av bruker}
% Etter autentiseringen er prøvde det å registrere ny bruker. Det oppsto en feilmelding under registrering av ny bruker
% \begin{lstlisting}[language=PHP]
%     SQLSTATE[HY000]: General error: 1364 Field 'image_id' doesn't have a default value
% \end{lstlisting}
% Løste feilen ved å endre standard verdien til "imageid" feltet i bruker database tabellen. Gjorde slik:

% \begin{lstlisting}[language=PHP]
%     $table->uuid('image_id')->nullable()->default(null);
% \end{lstlisting}

%--- Bjørnar - slutt ---

%--- Bereket - start ---

% \subsection{Roller og Permisjoner}\cite{spatie2019aupar}
% Laravel permisjon er en pakke med tillatelser og roller. Dette hjelper brukere til å knytte seg til tillatelser og roller. Hver rolle er knyttet til flere tillatelser. En rolle og en tillatelse er vanlige Eloquent-modeller\cite{oki2017uail}.
% \subsubsection{Laravel permisjon pakke installasjon}
% Laravel permisjon pakken er bygget ut over Laravel autorisasjon funksjoner\cite{laravelnews2017tblp}.
% For installere permisjon pakken kjørte composer require spatie/laravel-permission i kommando linje.

% \subsubsection{Inkludere pakken på service provider listen}
% Inkluderte permisjon pakken på config/app.php.
% \begin{lstlisting}[language=PHP]
%     Spatie\Permission\PermissionServiceProvider::class
% \end{lstlisting}

% \subsubsection{Publisere migrasjonen}
% Ved å kjøre koden på kommandolinjen publisere migrasjon filen for permisjon pakken.
% \begin{lstlisting}[language=PHP]
%   php artisan vendor:publish --provider="Spatie\Permission\PermissionServiceProvider" --tag="migrations"
% \end{lstlisting}

% Siden vi bruker uuid som primær nøkkel på user måtte det redigeres permisjon filen. På permisjon filen er satt \q{unsignedBigInteger} som standard og burde det endres til uuid. Redigeringen gjøres ved å åpne permisjon filen  under \q{database/migrations} og erstatte \q{unsignedBigInteger} med \q{uuid}.
% \begin{lstlisting}[language=PHP]
%   $table->unsignedBigInteger($columnNames['model_morph_key'])
%     Erstatter med
%  $table->uuid($columnNames['model_morph_key'])
% \end{lstlisting}

% \subsubsection{Publisere konfigurasjonen}
% Konfigureringsfilen tillater oss å angi plasseringen av Eloquent-modellen til permisjon og rolle klasse.
% For å publisere konfigurasjonsfilen for pakken kjøres koden nede i kommandolinjen.

% \begin{lstlisting}[language=PHP]
%   php artisan vendor:publish --provider="Spatie\Permission\PermissionServiceProvider" --tag="config"
% \end{lstlisting}

% \subsubsection{Bygge opp tabeller}
% Tok migrasjon slik permisjon tabellene blir bygget på databasen.
% Kjørte koden i kommandolinjen.
% \begin{lstlisting}[language=PHP]
%   php artisan migrate
% \end{lstlisting}

% For en bruker skal logge seg inn kreves det epost bekreftelse.
% Fikk feilmelding om mail bekreftelsen.
% \begin{lstlisting}[language=PHP]
%     Swift_TransportException in AbstractSmtpTransport.php line 383: Expected response code 250 but got code "530", with message "530 5.7.1 Authentication required
% \end{lstlisting}

% \subsection{Lage API}
% For å lage api opprettet det først controller\cite{laravel2019c} via cmd terminalen.
% Gjøres slik:
% \begin{lstlisting}[language=PHP]
%     php artisan make:controller PageController --resource
% \end{lstlisting}
% Den siste parameteren \q{resource} er et alternativ og kan dropes.

% På api.php i routes\cite{laravel2019r} registrer det API-ruter.
% For alle pages:
% \begin{lstlisting}[language=PHP]
%     Route::get('/pages', 'PageController@index');
% \end{lstlisting}
% For en enkel page:
% \begin{lstlisting}[language=PHP]
%     Route::get('pages/{page}', 'PageController@show');
% \end{lstlisting}

% \subsection{API Test}
% Brukte en verktøy som heter Postman for api test. Postman er en funksjonsrik REST-klient\cite{Rathod2017ITP}. Installerte postman på datamaskinen til å teste api-er. Postman kan installeres her\cite{postman2019tocap}
% \subsubsection{Utfordring under testing}
% På mange til mange relasjoner ble  det laget egne modeller på pivot tabellene.
% Forholdene ble satt opp via pivot tabellene sine modeller. På kontroller ble det brukt  Apend til å få alle forholdene mellom tabellene.
% \begin{lstlisting}[language=PHP]
%     public function menu_links(){
%          return $this->hasMany('App\MenuLink')
%     }
% \end{lstlisting}
% Da det ble testet om for eksempel en page ble brukt i menu tabellen via fremmednøkkel, fikk ikke svar.
% Fikset ved å sette opp forholdene ved hjelp av 'hasManyThrough'. Gjøres slik:
% \begin{lstlisting}[language=PHP]
%     public function links(){
%          return $this->hasManyThrough(
%             'App\Link',
%             'App\MenuLink',
%             'menu_id',
%             'id',
%             'id',
%             'link_id'
%          );
%       }
% \end{lstlisting}

% \subsection{Seeds}
% For å lage forskjellige bruker roller lagte det RoleSeeder i databasen.
% Kjørte koden i kommandolinjen for å opprette det.
% \begin{lstlisting}[language=PHP]
%     php artisan make:seeder RoleSeeder
% \end{lstlisting}
% Lagte forskjellige roller i RoleSeederen
% \begin{lstlisting}[language=PHP]
%     $user = Role::create(['name' => 'user']);
%   $moderator = Role::create(['name' => 'moderator']);
%   $admin = Role::create(['name' => 'admin']);
%   $superadmin = Role::create(['name' => 'superadmin']);
% \end{lstlisting}
% \subsection{CRUD}
% Opprettet CRUD (Create, Read, Update and Delete) filer for å opprette, lese oppdatere og slette ressurser. På ressurs kontroller kan skrives logikken om hvordan ressursene opprettes, lagres, leses, oppdateres og slettes.

% \lstinputlisting[caption={Laravel controller}, language=php]{laravel-code-examples/controller.php}

% Etter det ble ferdig laget med CRUD, testet om det fungerer.
% I create.blade.php filen prøvde å lage nedtrekksmeny med komponenter.
% Fikk feilmeldingen:
% \begin{lstlisting}[language=PHP]
%     "Property [component_id] does not exist on this collection instance.
% \end{lstlisting}

% Grunnen var at det ble hentet flere komponenter fra databasen og prøvde å sette de som en singel komponent.
% Løste feilen ved å bruke for løkke.
% \begin{lstlisting}[language=PHP]
%     @foreach($components as $component)
%     <option value="{{$component->id}}" {{old('component_id',component->id)}}? selected> {{$component->name}} </option>
%      @endforeach
% \end{lstlisting}

% \subsection{Route}
% Web.php fil i routes katalogen definerer  ruter  for webgrensesnitt\footnote{\url{https://laravel.com/docs/5.8/routing}}.
% Skrives for eksempel slik, hvis man har flere ressurser :
% \begin{lstlisting}[language=PHP]
%     Route::resources([
%         'pages' => 'PageController',
%         'components' => 'ComponentController',
%         'links' => 'LinkController',
%         'fields' => 'FieldController',
%         'menus' => 'MenuController'
%         ]);
% \end{lstlisting}

% I tillegg ble det lagte php filer for grensesnitt til ressurser. index.blade.php, create.blade.php, edit.blade.php, og show.blade.php
% \cite{savani2018lcrud}

% Tilgangene til ressurser kan være forskjellige fra bruker til bruker og gis for eksempel slik via kontrolleren.
% \begin{lstlisting}[language=PHP]
%     public function __contruct(){
%       $this->middleware('auth');
%       $this->middleware('role:superadmin');
%     }
% \end{lstlisting}

% Eksemplet viser at tilgangen er gitt bare til brukere med superadmin rolle.

% Har lagt linker for Menus og Pages til brukere som er logget seg inn.
% Etter linkene ble lagt og rollene blitt gitt fikk ikke det å åpne linkene eller registrere nye brukere. Fikk feilmelding om det ikke eksisterer rolle klasse.
% \begin{lstlisting}[language=PHP]
%     "Class App\Http\Spatie\Permission\Models\Role does not exist"
% \end{lstlisting}

% Prøvde å rette feilen ved å legge til rolle klasse i kernel.php slik:
% \begin{lstlisting}[language=PHP]
%   protected $routeMiddleware = ['role'=>Spatie\Permission\Models\Role::class];
% \end{lstlisting}
% Løsningen ga en annen feilmelding om klasse rolle ikke finnes
%  \begin{lstlisting}[language=PHP]
%       "Class 'Role' not found
%  \end{lstlisting}

% Feilen var at det ble lagt Role modelle klasse istedet Middleware klasse og løste ved å sette riktig Middleware klasse. Middleware \cite{laravel2019mw} er en mekanisme som kan  filtrere http-forespørsler. Den kan for eksempel verifisere en bruker er autentisert.
% \begin{lstlisting}[language=PHP]
%   'role' => \Spatie\Permission\Middlewares\RoleMiddleware::class,
% \end{lstlisting}

% Registering av en ny bruker gikk men fikk feilmelding samtidig.
% \begin{lstlisting}[language=PHP]
%   "Argument 1 passed to Illuminate\Auth\SessionGuard::login() must implement interface Illuminate\Contracts\Auth\Authenticatable, null given,
% \end{lstlisting}

% Feilen var at den prøver å sende den registrert brukeren til en side mens en annen  bruker er logget seg inn på siden. Registrering av en ny bruker gjennomføres av innlogget brukere med admin rettigheter.

% Løste ved å overskrive og slette koden som sender brukeren til siden. Slik:
% \begin{lstlisting}[language=PHP]
%   public function register(Request $request)
%       {
%           $this->validator($request->all())->validate();

%           event(new Registered($user = $this->create($request->all())));

%         Sletter dette linje
%             $this->guard('web')->login($user);

%           return $this->registered($request, $user)?: redirect($this->redirectPath());
%       }
% \end{lstlisting}

% \cite{spatie2019aupar}

%--- Bereket - slutt ---  
\clearpage

%--- Bjørnar kladd - start ---
%%%% NOTATER %%%
% Skrive om forutsetninger?
% Vi lager et headless cms.
% Visualisere prosjektet: https://github.com/acaudwell/Gource

\clearpage


\clearpage
%--- Bjørnar kladd - slutt ---

\section{Front-end}
Front-end ble i hovedsak gjort av Line, med god hjelp av Bjørnar. 

\subsection{Oppsett av React og tilhørende verktøy}

Det første som ble gjort i denne prosessen var å laste ned Node.js\footnote{\url{https://nodejs.org/en/}}. I dette prosjektet blir ikke Node brukt som en back-end tjeneste eller server, men for å kunne kjøre alle verktøy som må til for å kunne utvikle lokalt på datamaskinen. Det var den nyeste versjonen (11.9.0) som ble lastet ned. Deretter ble nettleserutvidelsen React Developer Tools lagt til fra  webstoren til Google Chrome. Det ble også opprettet et repository på Github og lastet ned på maskinen ved hjelp av GitKraken. 
Videre var det nødvendig med et kommandolinjevindu og en teksteditor, valget falt da på kommandolinjevinduet Powershell og teksteditoren Visual Studio Code.

For å sette opp selve React, ble  guiden til react.org for å sette opp en ny React App\footnote{\url{https://reactjs.org/docs/create-a-new-react-app.html\#create-react-app}} fulgt. Her ble det oppgitt at følgende kommandoer må kjøres for å opprette et nytt prosjekt:
\begin{lstlisting}[caption={Oppretting av React App},language=bash]
npx create-react-app my-app
cd my-app
npm start
\end{lstlisting}

Det oppstod problemer når kommandoen \q{npx create-react-app my-app} skulle kjøres, og det kom flere og lange feilmeldinger. Dette måtte undersøkes nærmere, og etter et Google-søk på deler av feilmeldingen ble det oppdaget en sak på Github som omhandlet samme problem\footnote{\url{https://github.com/facebook/react/issues/11933}}.
Løsningen var å installere yarn og create-react-app globalt med yarn. La så til følgende i windows PATH så kommandoen kunne kjøres:
\begin{lstlisting}
C:\\Users\Datahjelpen AS\AppData\Local\Yarn\bin
\end{lstlisting}
Dette løste problemet og prosessen kunne fortsette. Da ble følgende gjort:

\begin{itemize}
\item Kjørte \q{npm install} i mappen for prosjektet.
\item Kjørte \q{npm start} for å starte react serveren.
\end{itemize}

Etter disse stegene begynte planleggingen av de forskjellige komponentene.

\section{Planlegging av komponenter}
Komponenter i React kan beskrives som gjenbrukbare biter av kode. En stor fordel med komponenter er at de er mulig å gjenbruke, slik at det ikke lenger er nødvendig å skrive den samme koden flere ganger. 

Ved å se på den godkjente mockupen ble følgende komponenter fastsatt: 
\begin{itemize}
\item Meny
\item Header
\item Resultater
\item Prosess
\item Steg i prosess
\item Kontaktskjema
\item Footer
\item Ikon + link
\item Ikon + tekst
\item Hvit boks under header. ActionBox
\item Actions til actionsbox
\end{itemize}

\section{Oppretting av components}
Prosessen med oppretting av komponenter startet med å lage en samlemappe som ble kalt \q{components}, som ligger i mappen src. For hver komponent ble det laget en ny undermappe som inneholdt koden til selve komponenten og eventuelt tilhørende stilark.

\section{Ruting (Routing)}
Ruting er et mønstergjenkjenningssystem som tar seg av innkommende forespørsler og sender de til spesifikke kontroller-funksjoner. Ruting er ikke innebygd i React. Det ble derfor laget en Router.js-fil i components-mappa. Etter dette ble følgende kommando kjørt:

\begin{lstlisting}[caption={Installering av React ruting},language=bash]
npm install react-router-dom
\end{lstlisting}

Importerte deretter react-router-dom i filen Router.js

\begin{lstlisting}[caption={Importering av react-router-dom},language=bash]
import {BrowserRouter, Route, Switch} from "react-router-dom";
\end{lstlisting}

Den foreløpige versjonen av back-end ble lastet ned og lagret lokalt, slik at det ble mulig å teste underveis.

For å kunne snakke og bruke foreløpig backend ble Axios\footnote{Se mer i avsnitt \ref{sec:tool:axios}} benyttet. Hvis etableringen av forbindelsen er vellykket, slik at det har blitt opprettet kontakt med API via Axios, vil alle svar som ble gitt av back-end bli lagret i en array. 

Deretter ble det satt opp en dynamisk ruting, ved hjelp av en løkke som går igjennom alle eksisterende sider og setter opp linker til disse.

\section{Installering av Axios}
Axios blir i vårt system brukt til å sende forespørsler til back-end. Dette biblioteket ble installert ved å kjøre kommandoen under.  Kommandolinjevindu må åpnes i filstilen der mappa til prosjektet er plassert.
\begin{lstlisting}[caption={Installering av Axios},language=bash]
npm install axios
\end{lstlisting}


\section{Page components}

Det ble opprettet et objekt som inneholder alle komponentene som er definert i databasen. Ved å loope igjennom alle komponentene til en side, ble det da mulig å sjekke om navnene samsvarte. Hvis de gjorde dette vil det kjøres en løkke som går igjennom alle feltene en komponent inneholder. Til slutt blir alle komponentene til en side, med riktige props\footnote{Props i React brukes til å sende med utvalgt data til en komponent} skrevet ut.  

Før det ble opprettet et objekt som inneholdt alle eksisterende components, ble det forsøkt å kun hente ut angitt komponenter til siden i back-end og så opprette components ut av dette. Da fikk vi følgende feil:
\begin{lstlisting}
<HeaderComponent /> is using incorrect casing. Use PascalCase for React components, or lowercase for HTML elements.
\end{lstlisting}

Feilmeldingen var at komponenten brukte feil casing, noe som ikke var riktig da man i feilmeldingen også kunne se at \q{\textless HeaderComponent /\textgreater} er skrevet på riktig måte. Etter en del feilsøking, fant vi ingen som hadde løst dette problemet tidligere. Endte derfor opp med å opprette objektet som inneholder alle komponentene. 

I tillegg til å finne ut av hvilke komponenter som tilhører siden, må man i tillegg finne ut av hvilke felter som tilhører hvilke komponenter. Dette blir satt i back-end. I løkka som går igjennom objektene ble det derfor lagt til en ny løkke, som går igjennom og lagrer alle tilhørende felter til en komponent i en array. Denne arrayen blir så sendt videre som props i de ulike komponentene.

\section{Child components}
I systemet vårt kan et komponent ha flere \q{barn}. Det å ha komponenter som er barn av en komponent gjør at disse har et forhold til hverandre. Da blir det mulig å putte forskjellig innhold inn i en felles seksjon.

\section{Felter med samme navn}
Underveis i prosessen ble det oppdaget at hvis en komponent hadde den samme felttypen flere ganger, ville bare den siste verdien bli skrevet ut. En komponent vil ofte bestå av for eksempel flere tekstfelt, og derfor måtte vi finne en måte å løse dette på.

Løsningen ble en betingelsetest der det blir sjekket om samme felttype er til stede flere ganger. Hvis denne blir sann, vil feltene med verdier bli lagt til i en array. 

\section{Utskrift av tomme felter}
Videre i prosessen ble det oppdaget at hvis en prop ikke innholdt en verdi, ville feltet fortsatt bli skrevet ut. Dette førte til at nettstedet inneholdt mange tomme HTML-tagger. Brukervennligheten på for eksempel skjermlesere hadde blitt betraktelig dårligere, så derfor måtte dette gjøres om. Løsningen ble å gjøre en betingelse-test på hver prop, for å sjekke om den innholdt en verdi. Da ble det mulig å kun skrive ut om dette var oppfylt.

\section{Testing}
For å teste systemet ble header-komponenten laget. Når denne ble opprettet både i objektet og back-end, ble alle props/feltene som blir mottatt fra komponenten skrevet ut. Dette betydde at Header-komponenten fungerte som den skulle, og vi kunne starte med å legge til resten av komponentene på den samme måten. 

Da kunne vi også begynne å style slik komponentene slik at de ligner mockupen som ble laget.

\section{Kompilere SCSS til CSS}
Gruppen har tidligere nevnt at det ble besluttet å bruke SASS til styling. Da er det nødvendig å ha noe som kompilerer fra SCSS til CSS. Dette ble oppnådd ved å følge guiden hos TechCookbook\footnote{\url{https://techcookbook.com/react/use-scss-with-create-react-app}}

Installerte først node-sass ved å kjøre :

\begin{lstlisting}[caption={Installering av node-sass},language=bash]
npm install --save node-sass
\end{lstlisting}

Etter at pakken ble installert, måtte det lages et script som kompilerer fra .scss til .css:

\begin{lstlisting}[caption={Kompliering fra .scss til .css}]
 "build-css": "node-sass src/ -o src/"
\end{lstlisting}
\clearpage

\q{Build-css} tar .scss-filene som finnes i source-mappen samt undermapper og kompilerer dem til .css-filer. .css filen vil bli tilgjengelig i den samme lokasjonen som den originale .scss-filen. Deretter ble det laget et script som kjører \q{build-css} og kompilerer alle eksisterende filer til .css, samtidig som den ser etter forandringer i source-mappen. Den vil altså detektere om det blir gjort endringer i eksisterende .scss filer, eller om det blir lagt til nye. 

\begin{lstlisting}[caption={Script som detekterer scss endringer}]
 "watch-css": "npm run build-css && node-sass src/ -o src/ --watch --recursive"
\end{lstlisting}

En test ble gjort ved å lage en fil som het "header.scss" og satte bakgrunnsfargen til rød. Dette fungerte som det skulle.

\begin{lstlisting}
 body {
  background-color: red;
}
\end{lstlisting}

\section{Styling}
Designet til nettstedet ble stylet ved hjelp av SCSS, og er så lik den godkjente mockupen som mulig. Designet er også responsivt og fungerer derfor like godt på en stor skjerm, laptop, nettbrett og mobil. For stylingen ble prinsippet \q{mobil først} benyttet. Dette prinsippet går ut på at utviklingen først skjer for små skjermer, for deretter å skalere opp og gjøre designendringer når det er nødvendig. 


% \section{Utredning}

% \meta{
% Det er mulig at dette kapitellet er overflødig i et utredningsprosjekt. Utredningen er jo et eget dokument, og trenger vel ikke med kontekst for å kunne evalueres.
% }

% \section{Mediaproduksjon}

% \meta{
% For denne typen prosjekter kan det være relevant å beskrive og rapportere fra selve produksjonen. Det er vel relativt vanlig at man må endre og improvisere i forhold til opprinnelig plan, og det bør jo absolutt dokumenteres, ikke minst i forhold til diskusjonen (Kapittel 6.2). er mulig at dette kapitellet er overflødig i et utredningsprosjekt. Utredningen er jo et eget dokument, og trenger vel ikke med kontekst for å kunne evalueres.
% }

% \section{OpenOffice Writer}

% Se vedlegg i rapporten {\em OpenOffice mal for hovedprosjektrapport}.

% \section{\LaTeX}

% For et nærmere innblikk i hvordan denne malen er implementert, se kildekoden som følger med.
% Resultatet av den ønskede layout kan selvfølgelig sees i dokumentet du leser nå, eller i mer formell form i Figur \ref{fig:layout1}.


% \begin{figure} 
% \index{Recto} 
% \printinunitsof{mm}
% \currentpage
% \oddpagelayouttrue
% %\oddpagelayoutfalse 
% %\twocolumnlayouttrue 
% \pagediagram 
% %\drawpage
% %\pagedesign
% \center
% \setvaluestextsize{\scriptsize}
% \pagevalues
% \caption{Layout for recto sider} 
% \label{fig:layout1} 
% \end{figure}
