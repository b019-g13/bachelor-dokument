\cleardoublepage
\chapter{Introduksjon}
\label{chap:intro}
Kapittel 1 omhandler rammene rundt prosjektet, prosjektets formål metoder og leveranser. Her vil også oppdragsgiver, veileder og studengruppen bli presentert. 

\section{Om prosjektet}
Prosjektet er knyttet til Bacheloroppgave 2019, der oppdragsgiveren er Sirkus Media. Veileder for prosjektgruppen er Einar Krogh. 

\section{Prosjektgruppen}

Prosjektgruppen består av 3 medlemmer, der alle er dataingeniørstudenter. Gruppen ble allerede kjent på sommerkurset knyttet til Y-veien, og har tidligere samarbeidet på flere gruppeprosjekter gjennom hele studietiden. Allerede første år på studiet fant vi ut at vi samarbeidet godt, en faktor som var viktig for alle deltagerne på gruppen når det skulle dannes bachelorgruppe. Kompetansemessig utfyller gruppen hverandre godt, da alle har forskjellige områder og teknologier som foretrekkes å jobbe og lære mer om.  

\subsection{Om Bereket}
Gikk på IKT-servicefag på  videregående og var deretter lærling hos IT-avdelingen til Rana kommune. Etter bestått fagbrev gikk veien videre til Høgskolen i Østfold på dataingeniørstudiet, Y-veien.

Han er interessert i å jobbe med teknologier som HTML, CSS, PHP og JavaScript.

\subsection{Om Bjørnar}
Tok fagbrev som IKT-servicemedarbeider hos Optimale Systemer AS i Larvik etter 2 år som lærling. Dette etter å ha fullført IKT-linja hos Sandefjord videregående skole. Deretter gikk det videre til Høgskolen i Østfold, og startet da på Y-veien for dataingeniørstudiet.

Jobber nå som daglig leder hos Datahjelpen AS, hvor arbeidsoppgavene innebærer utvikling som full-stack og en del DevOps. Liker å jobbe med teknologier som HTML, CSS, JavaScript, Node, React, PHP og Linux.

\subsection{Om Line}
Gikk IKT-servicefag på Sandefjord videregående skole. Var etter dette lærling hos Vestfold Fylkeskommune der hun tok fagbrev og dermed ble utdannet IKT-servicemedarbeider. Startet deretter på Y-veien på dataingeniørstudiet ved Høgskolen i Østfold. Line har alltid vært interessert i data og teknologi, og skjønte tidlig at dette var noe hun ønsket å jobbe med.

For tiden jobber hun som front-end utvikler hos Datahjelpen AS og liker best å jobbe med teknologier som HTML, CSS, JavaScript og PHP.

\section{Oppdragsgiver}
Oppdragsgiver for dette prosjektet er Sirkus Media.
Sirkus Media ble stiftet i 2010 av Hans-Christian Hymer og er et teknologi-, data- og analyseselskap som leverer innovative løsninger for digital markedsføring. Deres kunder er små og store bedrifter i hele Norge som ønsker å få økt salg og lønnsomhet. 

Ved hjelp av deres egenutviklede løsning jobber de med å skaffe konkrete kundehenvendelser til sine kunder. Løsningen når direkte ut til kjøpsklare kunder. Dette innebærer at de utelukkende markedsfører mot personer som, basert på nettadferd, vet at med stor sannsynlighet er på utkikk etter produkter eller tjenester som selskapet tilbyr.

Sirkus Media har i dag 2 ansatte og hadde i 2017 en omsetning på 6.4 millioner\footnote{\url{https://www.proff.no/bransjes\%C3\%B8k?q=sirkus\%20media}}.
Vår kontaktperson hos oppdragsgiver er Hans-Christian Hymer som både er grunnlegger og daglig leder av Sirkus Media. 

\section{Oppdraget}
\label{sec:oppgaven}
Bakgrunnen for dette prosjektet er at Sirkus Media ønsker å forbedre sin profil på nett, slik at potensielle kunder får en bedre forståelse av deres produkter og tjenester og tar kontakt gjennom nettstedet. Dette er viktig for oppdragsgiver ettersom dagens nettsted har store mangler og ikke inneholder nok informasjon om hva Sirkus Media tilbyr. Dagens nettsted fungerer dermed ikke som den gode markedsføringskanalen den potensielt kunne vært. Sirkus Media har derfor et stort behov for utvikling av en ny nettside. 

Det er et krav at nettstedet skal følge gjeldende lover og regler i Norge, særlig med tanke på universell utforming og personvern. I tillegg bør nettstedet implementeres med beste praksis for søkemotoroptimlisering, struktur og semantikk. Det skal også være mulig å bruke nettstedet på de fleste type plattformer og enheter. Minstekravet er at nettsiden er tilpasset og funksjonell for mobil, nettbrett, MacOS, Linux og Windows.

I tillegg til nettstedet omfatter oppgaven: 
\begin{itemize}
\item Analysering av nåværende løsning, konkurrenter og passende verktøy
\item Utforming av logo, skisser, visuell profil og designsystem
\item Planlegging og oppretting av database
\item Analysering av ferdig produkt
\item Dokumentasjon og brukerveiledninger til oppdragsgiver
\end{itemize}

\section{Hvorfor, hva og hvordan: Formål, leveranser og metode}
\label{sec:maal-metode-resultater}
\subsection{Formål}
\label{sec:maal}
\begin{compactitem}
\item [{\bf Hovedmål}] Forbedre profilen til Sirkus Media på nett, slik at potensielle kunder får en bedre forståelse av deres produkter og tjenester og dermed tar kontakt gjennom nettstedet. På et overordnet plan vil dette bidra til å øke omsetningen til oppdragsgiver.
\begin{compactitem}
\item [{\bf  Delmål 1} ] Generere mer trafikk, som fører til at flere kunder tar kontakt med bedriften. 
\item [{\bf  Delmål 2} ] Måle trafikken på nettstedet, slik at man ser hva som fungerer og deretter kan tilpasse informasjonen til brukerne som besøker siden.
\end{compactitem}
\end{compactitem}

\subsection{Leveranser}
\label{sec:resultater}
Hovedresultatet til dette prosjektet vil være et komplett og offentlig nettsted, som er tilgjengelig på domenet til Sirkus Media.

% Bacheloroppgaven har i tillegg noen krav til leveranser og følgende dokumenter må derfor leveres:
%\begin{itemize}
%\item Prosjekt og gruppekontrakt
%\item Gruppenettsted
%\item Forprosjektrapport
%\item Første versjon av hoveddokument
%\item Andre versjon av hoveddokument
%\item Hoveddokument med vedlegg
%\item Prosjektplakat
%\item Presentasjon
%\end{itemize}

I tillegg til leveranser knyttet til bachelorprosjektet, vil disse konkrete resultatene bli opprettet underveis i prosjektperioden:
\begin{itemize}
\item Analyse av gammelt nettsted
\item Analyse av konkurrenter
\item Analyse av verktøy
\item Visuell profil og designsystem
\item Prototype og mockups av nytt nettsted
\item Database og modeller
\item Analyse av nytt nettsted
\item Sitemap
\item Brukerveiledninger
\end{itemize}

\subsection{Metode}
\label{sec:metode}
For å oppnå både hovedmål og delmål må gruppen gjøre noen tiltak. Hovedmålet, som beskrevet tidligere, er å forbedre nettprofil til Sirkus Media. Dette løser vi med å analysere dagens nettside og kartlegger hva som er bra og dårlig. Deretter går vi videre til å analysere konkurrentene sine nettsider og ser hvordan disse løsninger ser ut. Denne analyseringen kan gi oss inspirasjon til vår løsning, men også hint på hva vi ikke burde gjøre.

Det første delmålet som skal oppnås er å generere mer trafikk til nettsiden. Da må vi å lage et attraktivt nettsted som følger beste praksis for god søkemotoroptimalisering, semantikk og universell utforming. Disse tre faktorene er sterkt knyttet sammen, da både god semantikk og universell utforming vil forbedre søkemotoroptimaliseringen. God søkemotoroptimalisering er viktig når kundene skal finne en bedrift.

For å oppnå delmålet om måling av trafikk på nettstedet må vi benytte oss av verktøy som Google Analytics. Dette gir oss en komplett oversikt over trafikk og brukermønster.

Mockups oppnår vi ved å hente inspirasjon fra lignende nettsider og deretter lager et forslag på design til nettstedet. Før utviklingen av nettstedet begynner , må designforslaget blitt godkjent av oppdragsgiver.

Database og modeller løser vi ved å først tegne og deretter forbedre databasen ved hjelp av penn og papir. Deretter opprettes et UML-diagram digitalt.

\section{Rapportstruktur}

% \meta{
% Det er vanlig å avslutte innledningen med en oversikt over resten av rapporten, f.eks. slik som dette:
% }

% I Kapittel~\ref{chap:analysis} starter vi med å se på generelle krav til akademiske og tekniske dokumenter, og ikke minst hva som skiller disse fra ``vanlige'' dokumenter. Vi ser deretter nærmere på krav og retningslinjer til bachelorrapporter ved nasjonale og internasjonale læresteder. Vi går også gjennom HiØ/IT sine egen beskrivelse av hovedprosjektet.  Vi gir eksempler på maler fra andre læresteder. Deretter ser vi på de tekniske sidene ved å produsere store og komplekse dokumenter, med spesiell vekt på aktuelle programvareverktøy. Vi presenterer en overordnet design av vår rapportmal i Kapittel~\ref{chap:design}, og beskriver den konkrete implementasjon i OpenOffice i Kapittel~\ref{chap:implementation}. Løsningen blir ad-hoc evaluert i Kapittel~\ref{chap:evaluation}, og i Kapittel~\ref{chap:discussion} diskuterer vi resultatet av prosjektet. Rapporten avsluttes med en kort konklusjon i Kapittel~\ref{chap:conclusion}. En mer detaljert gjennomgang av hvordan malen kan brukes finnes i Vedlegg~ \ref{chap:how-to}.

I Kapittel~\ref{chap:analysis} starter vi med å se på relevant teori og beste praksis for planlegging, design og utvikling. Dette gjør at vi kan begynne å danne oss et bilde av hvordan et nettsted bør se ut. I Kapittel~\ref{chap:design} tar vi for oss hvordan vi faktisk har tenkt til å utforme nettstedet, og kommer til å være selve grunnlaget for utviklingen. Kapittel~\ref{chap:implementation} beskriver den konkrete implementasjonen av nettstedet. Deretter vil løsningen bli testet både teknisk, av utvalgte brukere og av oppdragsgiver i Kapittel~\ref{chap:evaluation}. I Kapittel~\ref{chap:discussion} diskuterer vi resultatet av prosjektet, og rapporten avsluttes med en kort konklusjon i Kapittel~\ref{chap:conclusion}.
