\cleardoublepage
\chapter{Design / Utforming/ Planlegging (Generisk tittel)}
\label{chap:design} 
\meta{
 Basert på beskrivelsen av oppgaven (Avsnitt \ref{sec:oppgaven}) og analysen i (Kapittel \ref{chap:analysis}), skal denne delen beskrive hvordan man har tenkt å utforme løsningen. Beskrivelsen er noe avhengig av type prosjekt.
 

\paragraph{Utredning}

Først og fremst må det redegjøres for hvordan man skal innhente informasjonen som skal danne grunnlaget for utredningen. Deretter må man bestemme hvordan man skal behandle materialet, f.eks. om man skal bruke statistiske metoder. Etter det er det naturlig å presentere en disposisjon for selve utredningen, som i dette tilfellet vil være hovedresultatet i prosjektet. Selve utredningen bør være et frittstående dokument, på samme måte som en film vil være et frittstående dokument i en mediaproduksjon, og et program vil være i et utviklingsprosjekt.

\paragraph{Mediaproduksjon}
Her vil det være naturlig å presentere story-boards og liknende metoder for å beskrive produksjonen. 

\paragraph{Utvikling}
Her er det bare å ta for seg av ulike måter å beskrive systemer på, alt fra overordnet arkitektur, moduler, meldingsprotokoller etc. Bruk gjerne formelle beskrivelsesspråk, typisk UML. 
}

I det videre tar vi for oss oppgaven med å utvikle struktur, innhold og formgiving av rapportmalen, uavhengig av hvilket verktøy som skal brukes ved implementasjonen.



---
\section{Strategi og planlegging}
I alle type prosjekter er det viktig å ha en strategi og planlegge etter dette. Kilde. Derfor kan det være helt avgjørende å ha en god plan.

Når det kommer til webdesign-prosjekter inneholder disse fire viktige komponenter. Første steg er at du må kjenne miljøet ditt. Andre steg er å planlegge. Tredje steg er å lære å kunne tilpasse seg, og siste steg er å fullføre prosessen ved å løse kompatibilitetsproblemer etter hvert som de oppstår. (Dawson, 2011).

Dawson, A. (2011). Future-proof web design. Retrieved from https://ebookcentral.proquest.com

\url{https://www.sciencedirect.com/science/article/pii/S014829631630203X#bb0175 }


---

\section{Struktur}

Vi tar utgangspunkt i den tradisjonelle og generiske strukturen beskrevet i Avsnitt \ref{sec:mayfield}. Det er etter min mening vanskelig å se noen grunn til å gjøre noe annet.

\section{Innhold}

Vi så Avsnitt \ref{sec:hiof-it}  at man ved HiØ skiller mellom produkt- og prosessdokumentasjon, og operer med flere frittstående dokumenter. Det er ønskelig at hovedokumentet omhandler løsningen av oppdraget, dvs. selve produktet.
Struktur og innhold bør være som som følger:

\begin{compactdesc}
\item [Front Matter]  Som i Avsnitt \ref{sec:mayfield}.

\item [Body] Hoveddelen av dokumentet (gjennomføringen).
\begin{compactdesc}
\item [Introduksjon] Skal gi leseren et bilde av rammene rundt prosjektet: Kort beskrivelse av oppdraget, litt om oppdragsgiver og prosjektgruppen, prosjektets formål, leveranser og metoder. Den bør også inneholde en oversikt over resten av rapporten. 
\item [Analyse] Kapittelet tar for seg analysedelen av arbeidet. Den består av to hoveddeler, en grundig beskrivelse av oppgaven basert på skissen gitt av oppdragsgiver, og en undersøkelse av hva som finnes av relatert arbeid, {\em best practise} og relevant teknologi. 
\item [Design]  Basert på beskrivelsen av oppgaven i introduksjonen  og analysen, skal denne delen beskrive hvordan man har tenkt å utforme løsningen. Beskrivelsen er nødvendigvis noe avhengig av type prosjekt. Designet er uavhengig hva slags verktøy man skal bruke i implementasjonen.
\item [Implementasjon] Her skal det beskrives hvordan man faktisk produserte leveransene i prosjektet, og viktigst, beskrive selve produktet. Hvilke verktøy brukte man, hvordan foregikk produksjonen, etc. Utformingen av dette kapittelet avhenger helt klart av type prosjekt.
\item [Evaluering] De fleste prosjekter avsluttes med en eller annen form for evaluering av resultatene. I utviklingsprosjekter vil det være naturlig med teknisk testing, fungerer programvaren som den skal? Teknisk testing kan utføres av utviklerne selv, eller en ekstern part, f.eks. oppdragsgiver. En oppdragsgiver ønsker ofte å utføre en akseptansetest, dvs. en test som vil avgjøre om de har "fått det de har betalt for". Ellers vil det i mange tilfeller være nyttig og viktig med en brukertest, dvs. en strukturert test der sluttbrukerne får komme til orde. 
\item [Diskusjon] Diskusjonskapittelet er viktig, både for dere selv og sensor. Dette kapitelet er det som i hovedsak skiller et akademisk prosjekt fra et rent næringslivsprosjekt. Ble resultatet som forventet? Ble oppdragsgiver fornøyd? Kunne dere gjort noe anderledes/bedre?
Det er her dere skal dokumentere at dere har lært noe underveis, ikke bare levert et produkt til en oppdragsgiver. 
\item [Konklusjon] Konklusjon er på et vis et sammendrag av diskusjonskapittelet. Her bør dere legge vekt på de viktigste funnene, både når det gjelder produktet og prosessen. Med et fyldig diskusjonskapittel bør trenger ikke konklusjonen bør være mer enn en side.
\end{compactdesc}

\item [Back Matter] Som i Avsnitt \ref{sec:mayfield}.

\end{compactdesc}

\section{Utformingen}

Det følger med en del standard, ferdig definerte stiler med enhver \LaTeX~distribusjon. Disse er gjennomprøvet over flere 10-år, og har tålt tidens tann både teknisk og estetisk. Når man skal lage sine egne design, vil det som regel alltid være fornuftig å gå ut i fra en  av standardstilene. Det gjør vi også i dette prosjektet, og tar utgangspunkt i stilen {\em report}, parametrisert for dobbeltsidig A4-format, med 11pt basis fontstørrelse. Vi foreslår følgende endringer i report-stilen: 

\paragraph{Marger}
I følge egen og kollegers erfaring er marginene noe store i standardrapporten, slik at det blir litt ``trangt'' for tekst, tabeller og illustrasjoner. Vi foreslår en layout med reduserte marger på alle sider.


\paragraph{Topptekst / bunntekst}
Når man slår opp en dobbeltsidig bok, kalles venstre side for {\em verso}, og høyre for {\em recto}.
\index{Recto}
Vi ønsker at bunnteksten skal være uten andre elementer enn fotnoter. Sidetallet bør trykkes i toppteksten, til venstre på verso sider, og til høyre på recto sider. På denne måten
blir det lett å bla fort igjen for å finne et gitt sidetall,  som f.eks. er funnet i en indeks. Videre er det ønskelig at aktuelt kapittel er angitt til høyre på verso sidene, og aktuelt underkapittel til venstre på recto sidene.

\paragraph{Tittelsiden} 
Det skal være to tittelsider, først en fengende forside, som studentene utformer selv, og deretter en standard bacheloroppgave-side som er lik for alle.


\section{Leveransene og malen}
Første versjon av hoveddokumentet  bør bestå av introduksjonen og analyse- og designkapitlene (Kapittel \ref{chap:intro}, \ref{chap:analysis} og Kapittel \ref{chap:design}), og andre versjon bør være en mer eller mindre komplett beskrivelse av selve resultatet (Kapittel \ref{chap:implementation}). Endelig leveranse tilsvarer den komplette rapporten pluss selve produktet (og poster og presentasjon).





