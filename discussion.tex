\cleardoublepage
\chapter{Diskusjon}
\label{chap:discussion} 
% \meta{
% Diskusjonskapittelet er viktig, både for dere selv og sensor. Dette kapitellet er det som i hovedsak skiller et akademisk prosjekt fra et rent næringslivsprosjekt. Det er her dere skal dokumentere at dere har lært noe underveis, ikke bare levert et produkt til en oppdragsgiver. 
% Her går dere tilbake til Avsnitt \ref{sec:maal-metode-resultater}. I hvilken grad ble målene oppnådd
%  (Avsnitt \ref{sec:maal})? Leverte dere de forventede resultatene (Avsnitt \ref{sec:resultater})? Fungerte metoden
%  dere beskrev i  Avsnitt \ref{sec:metode}?
% Hva ble bra? Hva ble ikke fullt så bra? I begge tilfeller, hvorfor? Hva slags problemer støtte dere på? Hva ville dere gjort anderledes, sett i ettertid? 
% }

I dette kapittelet vil vi presentere det vi har lært underveis. Her vil vi også se på om målene som ble satt i starten av prosjektet har blitt oppnådd. I tillegg skal vi presentere hva som har gått bra, hva som har gått dårlig og hva slags problemer som har oppstått underveis. Til slutt gjør vi en utredning på hva som burde ha blitt gjort annerledes, sett i ettertid.

\section{Mål som ble oppnådd}
I avsnitt \ref{sec:maal} definerte vi noen mål for prosjektperioden. Disse var som følgende:

\begin{compactitem}
\item [{\bf Hovedmål}] Forbedre profilen til Sirkus Media på nett, slik at potensielle kunder får en bedre forståelse av deres produkter og tjenester og dermed tar kontakt gjennom nettstedet. På et overordnet plan vil dette bidra til å øke omsetningen til oppdragsgiver.
\begin{compactitem}
\item [{\bf  Delmål 1} ] Generere mer trafikk, som fører til at flere kunder tar kontakt med bedriften. 
\item [{\bf  Delmål 2} ] Måle trafikken på nettstedet, slik at man ser hva som fungerer og deretter kan tilpasse informasjonen til brukerne som besøker siden.
\end{compactitem}
\end{compactitem}

Vi ser i ettertid at det ikke er mulig for oss å sjekke at delmålene har blitt oppfylt før etter endt bachelor, og de burde derfor blitt definert annerledes. Når det kommer til hovedmålet vil vi påstå at det er oppfylt. Vår løsning er mer innbydende, har et moderne utseende og gjenspeiler visjonen og bedriften bedre enn den tidligere løsningen. Vi har også en bedre presentasjon av bedriften, prosessen og deres tjenester som gjør det lettere for kundene å forstå hva oppdragsgiver tilbyr. Dette tror vi kommer til å bidra til at flere tar kontakt gjennom nettstedet.

\section{Det som har gått bra}
En av de tingene som har vært bra hele veien er samarbeidet i gruppa. Alle har vært arbeidsvillige og stått på det de kan. Selve fordelingen av arbeidsoppgaver har også fungert bra.

Ved oppstart valgte prosjektgruppen å kun møtes ved behov. Dette er en annen side som har fungert bra i denne perioden. 

\section{Problemer og utfordringer underveis i prosessen}
Vårt største problem underveis i prosessen var at vi hele veien var veldig tidsoptimistiske. Både utvikling og skriving har ofte tatt lenger tid enn forventet, som har ført til at vi ikke har rukket alt vi skulle. Planen var at vi skulle utvikle en visuell profil både med designprofil og designsystem. Vi skjønte etterhvert at vi ikke kom til å rekke alt vi hadde nevnt i leveranser, og valgte derfor å gå bort fra dette siden vi anså dette som den minst viktige leveransen. 

Det var også utfordrende at oppdragsgiver kun visste hva de ikke ville ha. Derfor tok det lengre tid å få mockupen godkjent enn det vi hadde regnet med, som igjen påvirket resten av prosjektperioden.

En annen utfordring har vært å sette seg inn i teknologiene vi ikke hadde erfaring med. Det tok lengre tid å sette seg inn i både Laravel og React enn vi i utgangspunktet trodde. 

\section{Dette burde ha blitt gjort annerledes}
Hvis vi skulle gjennomført dette prosjektet på nytt hadde vi utarbeidet en mer realistisk prosjektplan. 





