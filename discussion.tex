\cleardoublepage
\chapter{Diskusjon}
\label{chap:discussion} 

I dette kapittelet vil vi se på om metoden vi beskrev i introduksjonen fungerte, og om målene som ble satt i starten av prosjektet har blitt oppnådd. Vi skal også se på om forventet resultat ble levert. I tillegg skal vi presentere hva som har gått bra, hva som har gått mindre bra og hva slags problemer som har oppstått underveis. Til slutt gjør vi en utredning på hva som burde ha blitt gjort annerledes, sett i ettertid.

\section{Ble målene oppnådd?}
I avsnitt \ref{sec:maal} definerte vi noen mål for prosjektperioden. Disse var som følger:

\begin{compactitem}
    \item [{\bf Hovedmål}] Forbedre profilen til Sirkus Media på nett, slik at potensielle kunder får en bedre forståelse av deres produkter og tjenester og dermed tar kontakt gjennom nettstedet. På et overordnet plan vil dette bidra til å øke omsetningen til oppdragsgiver.
    \begin{compactitem}
        \item [{\bf  Delmål 1} ] Generere mer trafikk, som fører til at flere kunder tar kontakt med bedriften. 
        \item [{\bf  Delmål 2} ] Måle trafikken på nettstedet, slik at man ser hva som fungerer og deretter kan tilpasse informasjonen til brukerne som besøker siden.
    \end{compactitem}
\end{compactitem}

Vi ser i ettertid at det ikke er mulig for oss å sjekke at delmålene har blitt oppfylt før etter endt bachelor, og de burde derfor blitt definert annerledes. Eventuelt burde vi fått tilgang til domenet til oppdragsgiver tidligere, slik at vi hadde fått tid til å analysere trafikken. Planen var å benytte oss av verktøyet Google Analytics for å måle både trafikken og observere hvordan brukere benytter nettstedet. Da vi ikke fikk tilgang til oppdragsgiver domene, vet vi ikke om delmål 1 er oppfylt. Delmål 2 er ikke oppfylt, da vi ikke har satt opp noe verktøy for måling på nettstedet.

Når det kommer til hovedmålet vil vi likevel påstå at det er oppfylt. Vår løsning er mer innbydende, har et moderne utseende og gjenspeiler visjonen og bedriften bedre enn den tidligere løsningen. Vi har også en bedre presentasjon av bedriften, prosessen og deres tjenester som gjør det lettere for kundene å forstå hva oppdragsgiver tilbyr. Det er også lettere å ta kontakt gjennom det nye nettstedet, da kontaktskjemaet befinner seg flere steder, live-chatten er tilgjengelig på alle sider og både e-postadresse og telefonnummer til bedriften ligger lett tilgjengelig. Disse punktene tror vi kommer til å bidra til at flere tar kontakt gjennom nettstedet. 

I tillegg viser resultatet fra brukertestene at informasjonen om bedriften ligger lett tilgjengelig på nettsiden, og brukerne sitter derfor lettere igjen med en god forståelse av bedriften. Samtidig klarte alle testpersonen å kontakte bedriften gjennom en av de mulige måtene på nettstedet. Vi konkluderer derfor med at hovedmålet er oppfylt. 


\section{Fungerte metoden vi valgte?}
I avsnitt \ref{sec:metode} beskrev vi metoden vi hadde valgt. Så langt inn i prosjektperioden, vet vi at alle analysene og planlegging vi gjorde i starten av prosjektet fungerte bra. Dette ga oss et godt grunnlag for prosjektet, da vi fikk en oversikt over hvordan oppbyggingen til CMS-et og nettstedet burde se ut. 

For å oppnå delmålet om å generere mer trafikk til nettstedet, skulle vi ha stort fokus på SEO, semantikk og universell utforming. Dette har vi hatt under hele prosessen. Problemet er at vi ikke har lagt til verktøy som måler trafikk, og vi kan derfor ikke vite om denne metoden fungerte. Videre fungerte mockups og databaser bra, da det er lettere å utvikle når du vet hvordan database og design ser ut. Likevel ble ingen av delmålene ble oppfylt, og vi konkluderer med at vår utvalgte metode ikke fungerte optimalt.

\section{Prosjektets resultat}
Resultatet til prosjektet endte opp med å bli både et headless CMS og nettsted. Vi føler derfor at vi har levert utover forventet resultatet. Oppgaven gikk i hovedsak ut på å utvikle et nettsted, men vi har endt opp med å utvikle et eget headless CMS i tillegg til selve nettstedet. Vi ønsket å utfordre oss selv og sørge for at alle gruppemedlemmene fikk nok å gjøre, derfor ble oppgaven såpass stor. I tillegg har vi lagt stor vekt på at både CMS-et og nettsiden er responsivt, brukervennlig og at de følger beste praksiser nevnt i kapittel \ref{chap:analysis}. Videre har vi sørget for at nettstedet får gode resultater i de forskjellige testene vi har utført.

Planen var at vi skulle utvikle en visuell profil med både protype og designsystem. Vi skjønte etterhvert at vi ikke kom til å rekke alt vi hadde nevnt i leveransen, og valgte derfor å gå bort fra dette siden det ble ansett som det minst viktige. Utover dette har vi kommet i mål med alle leveransene som ble nevnt i forprosjektrapporten. Sitemap og brukerveiledning ligger vedlagt. Analyse av gammelt nettsted, konkurrent- og verktøyanalysen befinner seg i kapittel \ref{chap:analysis}. Visuell profil og mockups ligger i kapittel \ref{chap:design}, mens analysen av nytt nettsted befinner seg i kapittel \ref{chap:evaluation}. Videre ligger nettstedet og databaser med modeller/tabeller vedlagt på minnepenn. Nettstedet er også tilgjengelig på \url{https://sirkusmedia.b019-g13.group/}. CMS-et er tilgjengelig på \url{https://api.b019-g13.group/}.

\section{Det som ble bra}
Samtlige i gruppen er stolte over det ferdige produktet, og mener at resultatet både for CMS og nettstedet ble bra. Det er flere grunner til dette. Først og fremst syns vi at designet og brukervennligheten er god for back- og front-end. Resultatene fra brukertestene er med på å støtte vår påstand om at brukervennligheten for nettstedet er god. I brukertestene fikk vi også tilbakemeldinger på designet til nettstedet, som er med å støtter vårt utsagn om at designet er bra. I tillegg syns vi det meste av funksjonaliteten vi har lagt til i CMS-et er gjennomført og intuitivt. Vi har også fulgt best praksis på de områdene vi presenterte i kapittel \ref{chap:analysis}. Videre legger vi til rette for lover som universell utforming og GDPR. 

De tekniske testene viser at nettstedet presterer godt når det kommer til ytelse, universell utforming og SEO. Dette er andre faktorer som bidrar positivt.

Alle de foregående punktene er med på å vise hvorfor vi i studentgruppen syns det ferdige produktet ble bra.

\section{Det som ble mindre bra}
I CMS-et er vi mindre fornøyd med noe av funksjonaliteten. Det første er at om en komponent som har blitt brukt flere steder skal oppdateres, må den oppdateres på alle sider den blir brukt. Hvis for eksempel footeren skal oppdateres i dag, må den oppdateres på alle sider denne komponenten er på. Footeren skal i utgangspunktet være lik på alle sider, og det kan derfor føles unødvendig å måtte oppdatere alle.  Dette er grunnen til at denne funksjonaliteten bør legges til.

Videre fungerer ikke systemet helt optimalt når en link opprettes. Når en ny link har blitt lagt til, vil den ikke vises med en gang. Brukeren må oppdatere siden før den nye linken vises.

\section{Problemer vi har støtt på}
Det har oppstått problemer i utviklingsprosessen for både CMS-et og nettstedet. Disse blir beskrevet nærmere i kapittel \ref{chap:implementation}. Mot slutten av prosessen oppdaget vi at nettsiden ikke fungerte i Internet Explorer (IE). Grunnen til dette er at IE ikke støtter en del av JavaScript funksjonaliteten React krever\footnote{\url{https://facebook.github.io/create-react-app/docs/supported-browsers-features}}, samt biblioteket Axios, som vi har benyttet til API. Vi hadde fra før testet nettstedet i Edge, og antok at det også kom til å fungere i IE. Dette stemte dessverre ikke.

\section{Dette burde ha blitt gjort annerledes}
Hvis vi skulle gjennomført dette prosjektet på nytt, hadde vi utarbeidet en mer realistisk prosjektplan. Vi hadde også fulgt denne planen tettere. I tillegg hadde vi sørget for å oppdatere planen, når leveransene ikke ble ferdig til tiden vi hadde planlagt.

Som nevnt tidligere endte vi opp med å gjøre mye mer enn det oppdraget tilsa, og burde derfor vært flinkere til å begrense oss når det gjaldt tidsbruken av utvikling av brukergrensesnittet. Vi fant hele veien funksjonalitet vi ønsket å legge til for å forbedre brukeropplevelsen, og som ikke nødvendigvis var en del av oppgaven vi fikk i utgangspunktet. 

Vi burde også startet å teste nettstedet i andre nettlesere tidligere, slik at vi hadde hatt bedre tid til å feilsøke og rette feilen i Internet Explorer.

Det viktigste elementet vi hadde gjort annerledes hadde vært å laste opp nettstedet tidligere, slik at vi hadde fått oppfylt delmålene. 
