\cleardoublepage

\pagenumbering{roman} \setcounter{page}{1}
\chapter*{Sammendrag}

Sirkus Media ble stiftet i 2010 av Hans-Christian Hymer og er et teknologi-, data- og analyseselskap som leverer innovative løsninger for digital markedsføring. Deres kunder er små og store bedrifter i hele Norge som ønsker å få økt salg og lønnsomhet. 

Oppdraget går ut på å forbedre profilen til Sirkus Media på nett, slik at potensielle kunder får en bedre forståelse av deres produkter og tjenester og tar kontakt gjennom nettstedet. Dette er viktig for oppdragsgiver, ettersom dagens nettsted har store mangler og ikke inneholder nok informasjon om hva Sirkus Media tilbyr. Dagens nettsted fungerer dermed ikke som den gode markedsføringskanalen den potensielt kunne vært. 

Et nettsted kan nemlig ses på som et enveis onlineverktøy med høy bedriftskontroll. På en nettside har en bedrift mulighet til å bestemme innholdet og hvordan firmaet skal fremstå. Dette er en mulighet enhver bedrift burde benytte seg av, da nettstedet kan være med å forbedre og øke troverdigheten og dermed påvirke forholdet kunden har til bedriften.

I tillegg viser undersøkelser at det å benytte seg av internett har blitt et hverdagslig gjøremål for mange. Den første undersøkelsen kartla at hele 98\% av Norges befolkning hadde tilgang på internett hjemme i 2017. Videre viser en undersøkelse at hele 9 av 10 nordmenn surfet på nettet hver dag i 2017. SSB har også kartlagt  at så mange som 89\% av Norges befolkning, mellom 16-79 år, brukte internett til å søke etter informasjon om varer eller tjenester de siste tre månedene i 2018. Dette er svært relevante funn og er med å bevise hvorfor det er så viktig for en bedrift å være på internett i dag.

I dette prosjektet har vi valgt å benytte et CMS for å utvikle nettstedet, mer konkret et egenutviklet CMS basert på Laravel.

Rapporten gjennomgår hvordan vi har planlagt og utviklet CMS-et og nettstedet. I tillegg vil resultatene fra testing, evaluering og råd for eventuell videreutvikling av prosjektet bli presentert.