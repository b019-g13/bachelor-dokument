\cleardoublepage

\pagenumbering{roman} \setcounter{page}{1}
\chapter*{Sammendrag}

% \meta{
% Sammedraget er hele rapporten komprimert til max 1 side. Sammendraget skal gi leseren et godt og tilnærmet komplett bilde av innholdet i dokumentet. Akademiske sammendrag kalles på engelsk for ``Abstract'', og i mer kommersielle sammenhenger heter det gjerne ``Executive Summary''. I det siste tilfelle har sammendraget som hensikt å gi ledelsen i en bedrift nok informasjon til å ta økonomiske og/eller administrative avgjørelser\dots uten å lese hele rapporten (!). Tradisjonelt  blir sammendraget formattert som et sammenhengende avsnitt. I et bachelorprosjekt,
%  vil hovedformålet være å gi leseren (kanskje i første rekke sensor?) et informativt (og appetittvekkende) bilde av prosjektet. Det er ikke vanlig å bruke litteratur- eller kryssreferanser i sammendraget. Som en regel kan vi si at alt som står i sammendraget, kan det leses mer om i rapporten. Dermed blir utfordringen å belyse alle viktige hovedpunkter, kort og presist. For denne rapporten, kan det f.eks. bli som dette:
% }

% De nye retningslinjene for evaluering av bacheloroppgaver ved Høgskolen i Østfold/IT legger større vekt på hoveddokumentet enn før. Denne rapporten er resultatet av et prosjekt der formålet var å gi studentene en mulighet for å forenkle og forbedre dokumentproduksjonen. Rapporten er en selvforklarende mal som tar for seg innhold, struktur og layout av hoveddookumentet i bacheloroppgaven. I tillegg er den et konkret eksempel på hvordan man kan bruke \LaTeX~ som dokumentverktøy. Dokumentet er en mal, dvs. et stilsett som brukes for å gi dokumentet ønsket layout.  Det blir gitt eksempler på de viktigste teknikkene, slik som bruk av kryssreferanser, kildereferanser, figurer og tabeller, og eksempler på formattering av spesielle elementer, som lister, sitater, definisjoner og matematiske uttrykk. I de tilfellene eksemplene ikke er selvforklarende, blir det gitt råd om hvordan man skal få det til. Intensjonen er at malen kan brukes for alle de tre hovertypene av bachelorprosjekter ved HiØ/IT: Utredninger, mediaproduksjoner, og utvikling av programvare, maskinvare eller systemer. Der det er naturlig å differensiere innholdet i de enkelte kapitlene, blir det skissert mulige løsninger for alle typene prosjekt. Formgivingen er enkel, oversiktlig og tradisjonell. Utgangspunktet for strukturen er den generiske oppbyggingen av et teknisk-vitenskapelig dokument, slik det er beskrevet i {\em Mayfield Handbook of Technical \& Scientific Writing}.
% Innholdet i denne rapporten er en (kanskje forvirrende) blanding av generiske retningslinjer og konkret eksemplifisering relatert til prosjektet med å utvikle malen.

Sirkus Media ble stiftet i 2010 av Hans-Christian Hymer og er et teknologi-, data- og analyseselskap som leverer innovative løsninger for digital markedsføring. Deres kunder er små og store bedrifter i hele Norge som ønsker å få økt salg og lønnsomhet. 

Oppdraget går ut på å forbedre profilen til Sirkus Media på nett, slik at potensielle kunder får en bedre forståelse av deres produkter og tjenester og tar kontakt gjennom nettstedet. Dette er viktig for oppdragsgiver, ettersom dagens nettsted har store mangler og ikke inneholder nok informasjon om hva Sirkus Media tilbyr. Dagens nettsted fungerer dermed ikke som den gode markedsføringskanalen den potensielt kunne vært. 

Et nettsted kan nemlig ses på som et enveis onlineverktøy med høy bedriftskontroll. På en nettside har en bedrift mulighet til å bestemme innholdet og hvordan firmaet skal fremstå. Dette er en mulighet enhver bedrift burde benytte seg av, da nettstedet kan være med å forbedre og øke troverdigheten og dermed påvirke forholdet kunden har til bedriften.

I tillegg viser studier at det å benytte seg av internett har i dagens samfunn blitt et hverdagslig gjøremål for mange. Den første undersøkelsen kartla at hele 98\% av Norges befolkning hadde tilgang på internett hjemme i 2017. Videre viser en undersøkelse at hele 9 av 10 nordmenn surfet på nettet hver dag i 2017. SSB har også kartlagt  at så mange som 89\% av Norges befolkning, mellom 16-79 år, brukte internett til å søke etter informasjon om varer eller tjenester de siste tre månedene i 2018. Dette er svært relevante funn og er med å bevise hvorfor det er så viktig for en bedrift å være på internett i dag.

I dette prosjektet har vi valgt å benytte et CMS for å utvikle nettstedet, mer konkret et egenutviklet CMS basert på Laravel.

Rapporten gjennomgår hvordan vi har planlagt og utviklet CMS-et og nettstedet. I tillegg vil resultatene fra testing, evaluering og råd for eventuell videreutvikling av prosjektet bli presentert.