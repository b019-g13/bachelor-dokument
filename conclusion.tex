\cleardoublepage
\chapter{Konklusjon}
\label{chap:conclusion} 
% \meta{
% Konklusjon er på et vis et sammendrag av diskusjonskapittelet. Her bør dere legge vekt på de viktigste funnene. Med et fyldig diskusjonskapittel bør trenger ikke konklusjonen bør være mer enn en side, men legg vekt på tydelig og godt språk. Husk at sensor antagelig først leser sammedraget, deretter konklusjonen. Etter det står resten av dokumentet for tur.
% Fokuser på hvordan produktet ble i forhold til forventningene til oppdragsgiver og dere selv. Gjenta i kortform de viktigste punktene fra diskusjonskapitlet. I tillegg bør dere nå se videre, om hva som burde bli gjort ved en videreføring av prosjektet (framtidig arbeid). I tillegg kan dere her gi råd om videre arbeid.
% }

I dette kapittelet vil vi ta for oss hvordan det ferdige produktet ble i forhold til forventningene til oppdragsgiver og våre egne.  Vi skal også presentere hva som burde bli gjort ved en eventuell videreføring av prosjektet, og komme med råd til videre arbeid.

\section{Forventninger og det endelige resultatet}
Prosjektets hovedmål var å lage et nettsted som forbedrer profilen til Sirkus Media på nett, slik at besøkende fikk en bedre forståelse av bedriften og dermed øke sjansen for at de tar kontakt gjennom nettstedet. Dette føler samtlige på gruppen at vi har fått til. Som nevnt ved flere anledninger, føler vi at nettstedet nå fremstår som mer imøtekommende og profesjonelt, i tillegg til at det er responsivt og raskt.

Tilbakemeldingene på brukertestene var også veldig positive. Etter å ha rettet opp de punktene som fungerte mindre bra, tror vi at tilbakemeldingene hadde vært utelukkende positive om vi hadde gjort testene på nytt. 

De tekniske testene og sammenligningen mellom det nye og gamle systemet viser også det nye nettstedet presterer bedre med tanke på ytelse og hastighet, universell utforming og beste praksiser.

Til slutt har vi, gjennom tilbakemelding fra oppdragsgiver, fått inntrykk av de er svært positive til det ferdige produktet. Vi opplever derfor at både CMS og nettstedet står til forventingene til arbeidsgiver og våre egne.

\section{Videreføring av prosjektet}
Oppdragsgiver ønsket mulighet for å legge til ytteligere to sider, om behovet skulle melde seg. Dette tar systemet høyde for, og oppdragsgiver kan selv legge til disse ved å følge tilsendt brukerveiledning.

I utgangspunktet hadde vi også tenkt å laste opp CMS og nettsted på domenet til Sirkus Media, men vi fikk ikke tilgang til dette. I tillegg hadde vi ikke mottatt alt av innhold som skulle være på siden. Derfor valgte vi å laste det opp på midlertidige domener, slik at de var offentlig og kunne vises frem. Ved en videreføring av prosjektet ville det derfor vært naturlig å laste det opp på domenet til oppdragsgiver.

Ved videre arbeid burde CMS-et videreutvikles slik at en komponent med innhold kan brukes flere steder. Da blir det ikke lenger nødvendig å oppdatere samme komponent på alle sidene den befinner seg. Funksjonaliteten for å legge til en link burde også oppgraderes.

En annen faktor som bør legges til, er muligheten til å bruke nettstedet i Internet Explorer.

\section{Råd til videre arbeid}
Ved en eventuell videreføring av prosjektet vil vi anbefale å lese igjennom brukerveiledningen tilsendt oppdragsgiver, for å forstå logikken og hvordan systemet henger sammen. Da CMS-et er egenutviklet fra bunn av, finnes det heller ingen annen dokumentasjon enn denne bachelorrapporten og tilhørende vedlegg. Et godt råd vil derfor være å lese gjennom kapittel \ref{chap:implementation}, som omhandler gjennomføringen av prosjektet.

% Prosjektet har demonstrert at det er mulig å lage maler for hovedrapporten i bachelorprosjektet ved HiØ/IT i både OpenOffice og\LaTeX. Malene tilfredstiller kravene til struktur og utforming, og tilbyr nødvendig funksjonalitet. Det er imidlertid åpenbart at at \LaTeX~ er det mest robuste, fleksible, forutsigbare  og modne alternativet. OpenOffice egner seg relativt dårlig til kollaborativ skriving, fordi det er vanskelig å splitte dokumentet i biter, og vanskelig å gjennomføre en effektiv versjonskontroll. Hvis man likevel ønsker å bruke OO, anbefales det på det sterkeste at en enkelt person har ansvaret for dokumentet under hele prosjektet (resten av gruppa kan med fordel bidra, men da ved å forsyne ``brødtekst'' til den dokumentansvarlige).



