\clearpage

\section{Verktøy analyse}


\subsection{linux}
Linux er en åpenkilde operativsystem. Et operativsystem er programvare som administrerer alle maskinvareressursene som er tilknyttet skrivebordet eller datamaskinen. 
13

Linux har en rekke forskjellige versjoner kalt distribusjoner. De disribusjonene er Ubuntu Linux, Linux Mint, Arch Linux, Deepin, Fedora, Debian og openSUSE.

\url{http://lib.sgu.edu.vn:84/dspace/bitstream/TTHLDHSG/2808/1/Operating_System_Concepts_Essentials.pdf}

\subsection{nginx}
Nginx er et HTTP proxy program som kan håndtere høyere belastning av HTTP-forespørsler. 
Når nettstedet begynner å bli ferdig skal vi laste det opp på en server som kjører LEMP-stacken(Linux, Nginx, MariaDB og PHP). 
Vi valgte nginx for at den er raskere og klarer å håndtere høyere belastning sammenlignet med Apache ved hjelp av det samme settet av maskinvare.

\url{https://simplercloud.wordpress.com/2014/08/06/what-is-the-difference-between-lamp-stack-and-lemp-stack/}

\subsection{php}
\subsection{sessions}
\subsection{phpmyadmin}
\subsection{mariadb}
\subsection{mvc}
\subsection{symfony}
\subsection{laravel}
Siden oppdragsgiver har ikke satt noe krav til bruk av verktøy hadde vi fri om verktøy valg. Gruppen diskuterte om hvilke verktøy skulle brukes for å utføre prosjektet. Diskusjonen var om vi skulle bruke Laravel rammeverk eller Wordpress CMS (Control Management System). Begge verktøyene har sine gode og svake sider.  Etter vi så på begge verktøyene ble vi enige å bruke Laravel. Valget ble utført basert på informasjon som vi fant på nett og erfaring fra gruppemedlemmer. 

Rammeverket brukes vi for å utvikle et REST-API. 
Laravel er en åpen kilde PHP-rammeverk som brukes til å lage webapplikasjoner og andre typer nettsteder. Det er basert på Symfony. Symfony er et PHP webapplikasjonsramme og et sett med gjenbrukbare PHP-komponenter eller biblioteker. Det følger Model-view-controller (MVC) designmønsteret.  Dette er utgitt med innebygde funksjoner for autentisering, lokalisering, modeller, visninger, sesjoner, ruting og andre funksjoner som en inversjon av kontroll og et templerende system kalt Blade. Den nye funksjonen kommandolinje grensesnitt kalles artisan, innebygd støtte av database management system, støtte for håndtering hendelser og pakkesystem kalt bundler.

\url{https://www.educba.com/laravel-vs-wordpress/}

\subsubsection{Fordel}
\begin{itemize}
\item Laravel er et rammeverk og er veldig raskt.
\item Har mange funksjoner som autentisering, autorisasjon, inversjon av kontroll osv.
\item Alt kan tilpasses og styres hvordan tingene fungerer.
\item Databasen kan brukes eller utformes på egen måte.
\end{itemize}

\subsubsection{Ulemper}
\begin{itemize}
\item I Laravel må SEO definere egne ruter, og det tar mye arbeid å utvikle et nettsted som hovedsakelig er avhengig av innhold.
\item Laravel rammeverk er litt komplisert og krever mer kunnskap.
\item Laravel er mindre fleksibel for å oppdatere innholdet.
\end{itemize}

\subsection{csrf}
\subsection{CI \& CD}
\subsection{buddy.works}
\subsection{github}
 Git er et distribuert versjonskontrollsystem for sporing av versjoner av filer. Versjonskontroll er et system som registrerer endringer i en fil eller et sett med filer og man kan se endringene senere. Github er webportal for Git-repositoriene. Git lar deg spore og vert versjoner av filer på Github. Github hjelper utviklere for å samarbeide sammen offentlig eller privat.
 
\subsection{overleaf}
\subsection{ottomatik.io}

\subsection{amazon web services (AWS) S3}
\subsection{http2}
HTTP2 er en HTTP protokoll som gjør at vi trenger et SSL/TLS sertifikat for å servere nettsider over HTTPS. SSL/TLSsertifikatet kan skaffes gjennom Let’s Encrypt.  

\url{https://kinsta.com/learn/what-is-http2/}

\subsection{https}
HTTPS (Hypertext Transfer Protocol Secure) er en sikrere versjon av HTTP-protokollen som overfører data på web. HTTP er en applikasjonslagprotokoll for overføring av hypermedia dokumenter, for eksempel HTML. HTTPS-protokollen oppretter kryptert kanal mellom nettleseren og webserveren slik  kommunikasjonen blir sikkert. 

\url{https://www.rfc-editor.org/rfc/pdfrfc/rfc2616.txt.pdf}

\subsection{Let’s Encrypt}
Let’s Encrypt er en gratis og åpent sertifikatautoritet som gir X.509-sertifikater for kryptering av transportlagsikkerhet(TLS). Tjenesten leveres av Internet Security Research Group (ISRG).

\url{https://letsencrypt.org/}

\subsection{ssl/tls}
\subsection{html}
\subsection{css}
\subsection{sass}
\subsection{javascript}
\subsection{reactjs}
\subsection{axios}
\subsection{rest-api}
REST API-design (Representational State Transfer) er designet for å utnytte eksisterende protokoller. REST kan brukes over nesten hvilken som helst protokoll. Når den brukes til web-APIer benytter den vanligvis HTTP. Dette betyr at utviklere ikke trenger å installere biblioteker eller tilleggsprogramvare for å bruke REST API-design.

\url{https://www.mulesoft.com/resources/api/what-is-rest-api-design}

\subsection{Figma}
Figma er et grensesnitt design program som kjører i en nettleser. Vi skal bruke figma for å lage mockups. Med Figma har designere muligheter å dele en redigerbar design med andre. I tillegg kan designere også overføre filer til utviklere i en view-only modus. I denne modusen kan en utvikler inspisere design og eksportkode.

\url{https://www.xfive.co/blog/figma-best-designer-developer-cooperation/}

\subsection{ Google Analytics}
Google Analytics er en gratis webanalysetjeneste som gir statistikk og grunnleggende analytiske verktøy for søkemotoroptimalisering (SEO) og markedsføringsformål. Google Analytics hjelper for å måle, samle, analysere og rapportere av internettdata med hensikt å forstå og optimalisere bruken av nettet.
Når nettstedet er lansert skal vi måle trafikk og bruk av nettsidene ved hjelp av Google Analytics.

\url{https://searchbusinessanalytics.techtarget.com/definition/Google-Analytics}

\clearpage