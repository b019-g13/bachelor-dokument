\clearpage

\section{Verktøyanalyse}


\subsection{linux}
Linux er en åpenkilde operativsystem. Et operativsystem er programvare som administrerer alle maskinvareressursene som er tilknyttet skrivebordet eller datamaskinen. 

Linux har en rekke forskjellige versjoner kalt distribusjoner. De disribusjonene er Ubuntu Linux, Linux Mint, Arch Linux, Deepin, Fedora, Debian og openSUSE.

\url{http://lib.sgu.edu.vn:84/dspace/bitstream/TTHLDHSG/2808/1/Operating_System_Concepts_Essentials.pdf}

\subsection{nginx}
NGINX er en gratis og åpen kilde HTTP proxy server. Nginx kan håndtere høyere belastning av HTTP-forespørsler. Den er kompatibel med operativsystemer som Windows, Linux, Mac OS, FreeBSD og Solaris.

\url{https://simplercloud.wordpress.com/2014/08/06/what-is-the-difference-between-lamp-stack-and-lemp-stack/}


\subsection{php}
PHP er en forkortelse for Hypertext Pre-processor. PHP er et skriptspråk som er spesielt egnet for webutvikling og kan legges inn i HTML. PHP er en serverside programmeringsspråk derfor kan man programmere dynamiske, interaktive websider med PHP. PHP er gratis og plattformuavhengig. PHP er rask og fleksibel. Dette kan installeres i pakker sammen med web servere som LAMP, WAMP, XAMPP og LEMP.

\url{https://www.guru99.com/what-is-php-first-php-program.html}

\subsection{sessions}

\subsection{phpmyadmin}
phpMyAdmin er et gratis og web-basert administrasjonsverktøy for MySQL-database. Dette er skrevet i PHP. phpMyAdmin gjør flere operasjoner på MySQL og MariaDB. Med phpMyAdmin kan man lage og slette databaser, lage /slette /endre tabeller, slette/redigere/ legge til felt, utføre noen SQL-setninger, administrere nøkler på felt, administrere privilegier og eksporter data til ulike formater.

\url{https://www.phpmyadmin.net/}

\subsection{mariadb}

\subsection{mvc}

\subsection{symfony}

\subsection{laravel}
Laravel er en åpen kilde PHP-rammeverk som brukes til å lage webapplikasjoner og andre typer nettsteder. Det er basert på Symfony. Symfony er et PHP webapplikasjonsramme og et sett med gjenbrukbare PHP-komponenter eller biblioteker. Det følger Model-view-controller (MVC) designmønsteret.

Dette er utgitt med innebygde funksjoner for autentisering, lokalisering, modeller, visninger, sesjoner, ruting og andre funksjoner som en inversjon av kontroll og et templerende system kalt Blade. Den nye funksjonen kommandolinje grensesnitt kalles artisan, innebygd støtte av database management system, støtte for håndtering hendelser og pakkesystem kalt bundler.

\url{https://www.educba.com/laravel-vs-wordpress/}

\subsubsection{Fordel}
\begin{itemize}
\item Laravel er veldig raskt.
\item Har mange funksjoner som autentisering, autorisasjon, inversjon av kontroll osv.
\item Alt kan tilpasses og styres hvordan tingene fungerer.
\item Databasen kan brukes eller utformes på egen måte.
\end{itemize}

\subsubsection{Ulemper}
\begin{itemize}
\item I Laravel må SEO definere egne ruter, og det tar mye arbeid å utvikle et nettsted som hovedsakelig er avhengig av innhold.
\item Laravel rammeverk er litt komplisert og krever mer kunnskap.
\item Laravel er mindre fleksibel for å oppdatere innholdet.
\end{itemize}

\subsection{csrf}

\subsection{CI \& CD}

\subsection{buddy.works}

\subsection{github}
 Git er et distribuert versjonskontrollsystem for sporing av versjoner av filer. Versjonskontroll er et system som registrerer endringer i en fil eller et sett med filer og man kan se endringene senere. Github er webportal for Git-repositoriene. Git lar utvikleere spore og vert versjoner av filer på Github. Github hjelper utviklere for å samarbeide sammen offentlig eller privat.
 
 \url{https://searchitoperations.techtarget.com/definition/GitHub}
 
\subsection{Overleaf}
Overleaf er en gratis tjeneste som brukes til å lage, redigere og dele akademiske artikler enkelt på nettet ved hjelp av LaTeX. Dette er en nettbasert redigering og skriving program. 
Overleaf v2 tilbyr direkte synkronisering til GitHub og mulighet til toveissynkronisering mellom Overleaf og Dropbox.

\url{http://web.simmons.edu/~wilsonjd/LIS488/website/OverleafTutorial.pdf}

\url{https://no.overleaf.com/learn/how-to/Working_Offline_in_Overleaf}

\subsection{ottomatik.io}

\subsection{amazon web services (AWS) S3}

\subsection{HTTP2}
HTTP2 er en HTTP protokoll krever et SSL/TLS sertifikat for å servere nettsider over HTTPS. SSL/TLSsertifikatet kan skaffes gjennom Let’s Encrypt.  

\url{https://kinsta.com/learn/what-is-http2/}

\subsection{HTTPS}
HTTPS (Hypertext Transfer Protocol Secure) er en sikrere versjon av HTTP-protokollen som overfører data på web. HTTP er en applikasjonslagprotokoll for overføring av hypermedia dokumenter, for eksempel HTML. HTTPS-protokollen oppretter kryptert kanal mellom nettleseren og webserveren slik  kommunikasjonen blir sikkert. 

\url{https://www.rfc-editor.org/rfc/pdfrfc/rfc2616.txt.pdf}

\subsection{Let’s Encrypt}
Let’s Encrypt er en gratis og åpent sertifikatautoritet som gir X.509-sertifikater for kryptering av transportlagsikkerhet(TLS). Tjenesten leveres av Internet Security Research Group (ISRG).

\url{https://letsencrypt.org/}

\subsection{SSL/TLS}

\subsection{HTML}
HTML (Hypertext Markup Language) er et oppslagsspråk som brukes til å lage nettsidedokumenter. Det er et system som identifiserer og beskriver de forskjellige komponentene i et dokument som overskrifter, avsnitt og lister. HTML5 er den nyeste versjonen av HTML-standarden.

\subsection{CSS}
CSS (Casecading Style Sheet) beskriver utseendet på nettside som inneholder dokument, bilde, tekst eller annet som er utformet for å være mer attraktivt for brukere og hjelper dem med å identifisere hva de ser. CSS styrer fonter, farger, bakgrunnsbilder, linjeavstand, sidelayout osv. CSS3 er den nyeste versjonen av CSS og man kan legge til spesialeffekter og grunnleggende animasjoner til siden.

\subsection{SASS}
SASS (Syntactically Awesome Style Sheets) er en utvidelse av CSS. SASS tilbyr muligheter å bruke ting som variabler, nestede regler, inlineimport og mixins som gjør det lettere å gjenbruke CSS syntaks.. Det lar deg også lage stilark raskere. Sass er kompatibel med alle versjoner av CSS.  

\subsection{javascript}
Javascript er et programmeringsspråk som er mye brukt til å utvikle webapplikasjoner. Den støttes av de fleste av de moderne nettleserne. JavaScript brukes til å manipulere elementene på nettsiden og stilene som er brukt på dem. JavaScript støtter objektorientert språk.

\subsection{reactjs}
Reactjs er et fleksibelt JavaScript bibliotek for å bygge brukergrensesnitt. React er Det er visningslaget for webapplikasjoner. React er en åpen kildekode og utviklet av Facebook for å bygge brukergrensesnitt. Node.js er et JavaScript runtime miljø som gjør at vi kan kompilere React kode.

\url{https://www.fullstackreact.com/assets/media/sGEMe/MNzue/30-days-of-react-ebook-fullstackio.pdf}

\subsection{axios}

\subsection{rest-api}
REST API-design (Representational State Transfer) er designet for å utnytte eksisterende protokoller. REST kan brukes over nesten hvilken som helst protokoll. Når den brukes til web-APIer benytter den vanligvis HTTP. Dette betyr at utviklere ikke trenger å installere biblioteker eller tilleggsprogramvare for å bruke REST API-design.

\url{https://www.mulesoft.com/resources/api/what-is-rest-api-design}

\subsection{Figma}
Figma er et grensesnitt design program som kjører i en nettleser. Med Figma har designere muligheter å dele en redigerbar design med andre. I tillegg kan designere også overføre filer til utviklere i en view-only modus. I denne modusen kan en utvikler inspisere design og eksportkode.

\url{https://www.xfive.co/blog/figma-best-designer-developer-cooperation/}

\subsection{ Google Analytics}
Google Analytics er en gratis webanalysetjeneste som gir statistikk og grunnleggende analytiske verktøy for søkemotoroptimalisering (SEO) og markedsføringsformål. Google Analytics hjelper for å måle, samle, analysere og rapportere av internettdata med hensikt å forstå og optimalisere bruken av nettet.

\url{https://searchbusinessanalytics.techtarget.com/definition/Google-Analytics}

\subsection{Lighthouse}
Lighthouse er et open-source, automatisert verktøy for å forbedre kvaliteten på nettsidene. Du kan kjøre den mot en hvilken som helst nettside, offentlig eller krever godkjenning. Den har revisjoner for ytelse, tilgjengelighet, progressive webapps og mer. Den kan kjøres i Chrome DevTools, fra kommandolinjen, eller som en node modul.

\subsection{WAVE}
WAVE (Web Accessibility Evaluation Tool) er utviklet og gjort tilgjengelig som en gratis samfunnstjeneste av WebAIM. WAVE er en tilgjengelighetstest av nettsider.

\url{https://wave.webaim.org/about}
\clearpage