\clearpage

\section{Presentasjon av verktøy}

\subsection{Laravel}
Laravel\footnote{\url{https://laravel.com/}} er et PHP-rammeverk, med åpen kildekode, som brukes til å lage webapplikasjoner og andre nettsteder. Rammeverket baserer seg på Symfony\footnote{Avsnitt \ref{sec:tools-symfony}} og følger Model-view-controller\footnote{Avsnitt \ref{sec:tools-mvc}} (MVC) designmønsteret.
Laravel kommer med innebygde funksjoner for autentisering, modeller, visninger, sesjoner, ruting og andre funksjoner som en \textit{templating engine} kalt Blade\footnote{\url{https://laravel.com/docs/5.7/blade}}.

%%% VI SER PÅ FORDELER OG ULEMPER SENERE %%%
%\subsubsection{Fordeler}
%\begin{itemize}
%    \item Laravel er veldig raskt i forhold til CMS (Content Management Systems).
%    \item Har funksjoner som autentisering og autorisasjon.
%    \item Alt kan tilpasses og styres hvordan tingene fungerer.
%    \item Databasen kan brukes eller utformes på egen måte.
%\end{itemize}

%\subsubsection{Ulemper}
%\begin{itemize}
%    \item I Laravel må SEO definere egne ruter, og det tar mye arbeid å utvikle et nettsted som hovedsakelig er avhengig av innhold.
%    \item Laravel rammeverk er litt komplisert og krever mer kunnskap.
%    \item Laravel er mindre fleksibel for å oppdatere innholdet.
%\end{itemize}

%Fordeler og Ulemper av laravel ble tatt i forhold til CMS.
%Kilde: \url{https://www.educba.com/laravel-vs-wordpress/}

\subsection{Symfony}
\label{sec:tools-symfony}
Symfony\footnote{\url{https://symfony.com/what-is-symfony}} er et webapplikasjonsrammeverk og er et sett med gjenbrukbare komponenter og biblioteker skrevet i PHP.

\subsection{MVC}
\label{sec:tools-mvc}
Model-View-Controller(MVC)\footnote{\url{http://www.dgp.toronto.edu/~dwigdor/teaching/csc2524/2012_F/papers/mvc.pdf}} er et designmønster som deler et program inn i tre hovedkomponenter: modell, visning og kontroll. Oppbygningen kan enkelt forklares ved hjelp av et eksempel fra Matteo publisert på dev.to:
Matteo J. (2018, Mars 16) Re: Explain MVC like I'm five [Blogg kommentar]\footnote{\url{https://dev.to/matteojoliveau/comment/2ikd}}
\begin{quote}
    You walk in a McDonald and you order at the big interactive touchscreen.
    The screen is the View, and you ask for your meal there.
    The order is passed to the operator, who is the Controller, that will go in the kitchen and fetch a burger (your wanted resource) of the type (Model) you asked, then deliver it back to you.
\end{quote}
%Typisk flyt er at det kommer inn en forespørsel som går til kontrolleren. Den tolker forespørselen og behandler den ved å hente og endre data via modell. Til slutt sendes det et svar til visning.

\subsubsection{Modell}
Modell (Model) representerer de forskjellige type ressurser en applikasjon har. Her beskrives hver ressurs, hvordan den er oppbygd og hva som kan gjøres med den.

\subsubsection{Visning}
Visning (View) håndterer brukerinteraksjon og presentasjon av data. Er ofte et grafisk brukergrensesnitt.

\subsubsection{Kontrolleren}
Kontroll (Control) delen er ansvarlig for å koordinere forespørsler og svar.

\subsection{GNU/Linux}
GNU/Linux er en familie med Unix-lignende operativsystemer som baserer seg på Linux-kjernen og en del programvare fra GNU-prosjektet.
I denne familien finner vi blant annet Ubuntu, Debian, Fedora, Red Hat Linux og Arch Linux.
Kilder: \url{https://www.kernel.org} \url{https://www.gnu.org/} \url{https://www.ubuntu.com/} \url{https://www.debian.org/} \url{https://getfedora.org/} \url{https://www.redhat.com/en/topics/linux} \url{https://www.archlinux.org/}

\subsection{Nginx}
Nginx er en gratis HTTP-server med åpen kildekode. Nginx har mulighet til å håndtere høy belastning av HTTP-forespørsler. Programmet er tilgjengelig på operativsystemer som Windows, Mac OS og Solaris. I tillegg er Nginx tilgjengelig på operativsystemer som er basert på GNU/Linux eller BSD.

Kilde: Nedelcu, C. (2010). Nginx HTTP Server: Adopt Nginx for Your Web Applications to Make the Most of Your Infrastructure and Serve Pages Faster Than Ever. Packt Publishing Ltd.

\subsection{PHP}
PHP (rekursiv akronym for PHP Hypertext Pre-processor) er et skriptspråk som er spesielt egnet for webutvikling og kan legges inn i filer sammen med HTML. PHP er et serverside programmeringsspråk og kan brukes til å lage dynamiske og interaktive nettsteder. PHP er gratis og plattformuavhengig, samtidig som det er  er raskt og fleksibelt. Det kan installeres i pakker med webserver og database. Eksempelvis LAMP, LEMP, WAMP og XAMPP. PHP er også mulig å installere enkeltstående. 

Kilde: \url{http://php.net/}

\subsection{Sesjoner}
En sesjon\footnote{\url{https://ieeexplore.ieee.org/abstract/document/8392612}} (session) er en samling av data lagret på en webserver. En webserver tildeler en ID til hver bruker som sender forespørsler via en nettleser. ID-en lagres som en informasjonskapsel i nettlesern til brukeren. Alle nye forespørsel vil så identifiseres av ID-en som er lagret hos brukeren.

\subsection{phpMyAdmin}
phpMyAdmin\footnote{\url{https://www.phpmyadmin.net/}} er et gratis og webbasert administrasjonsverktøy for MySQL og MariaDB databaser. Verktøyet er skrevet i PHP og JavaScript. Med phpMyAdmin kan man lage, endre og slette databaser, tabeller og felter. Annen funksjonalitet kan være å utføre SQL-setninger, administrere nøkler og privilegier og eksportere data til ulike formater.

\subsection{MariaDB}
MariaDB er et relasjonsdatabasesystem basert på MySQL, som er gratis og har åpen kildekode. Systemet funker som en \textit{drop-in} erstatning for MySQL.

Kilde \url{https://mariadb.org/}

\subsection{CSRF}
Cross site request forgery\footnote{url{https://ieeexplore.ieee.org/abstract/document/5283085}} (CSRF) er angrep som tvinger brukere til å utføre uønskede handlinger på en nettside der brukeren er logget inn, via en annen nettside. For eksempel: hvis brukeren er logget inn på Facebook.com, og går til Twitter.com. Da kan Twitter laste inn et bilde som dette: \lstinline{<img src="https://facebook.com/delete-my-account">}. Dette vil da sende en forespørsel til den linken, som kan føre til sletting av konto. Om Facebook ikke har implementert mottiltak for slike angrep, vil denne forespørselen tolkes som om den kommer direkte fra brukeren.

\subsection{CI \& CD}
CI står for continuous integration og CD står for continuous delivery. CI/CD\footnote{\url{https://www.atlassian.com/continuous-delivery/principles/continuous-integration-vs-delivery-vs-deployment}} er en metode som lar utviklere implementere og levere kodeendringer raskt og pålitelig.

\subsection{Buddy}
Buddy\footnote{\url{https://buddy.works}} er en webbasert applikasjon for CI og CD. Applikasjonen lar utviklere bygge, teste og distribuere nettsider og programkode automatisk når kildekoden oppdateres. For eksempel er det mulig å koble Buddy til et nettsteds kildekode via Git. Når koden til nettstedet oppdateres, vil Buddy automatisk detektere endringer og starte en byggeprosess. Når byggeprosessen er ferdig kjøres det automatisk tester på koden, som verifiserer at alt fungerer. Deretter vil Buddy laste koden opp til en server og oppdatere nettstedet som ligger ute på nett. Dette skjer gjerne gjennom Docker\footnote{\url{https://www.docker.com/}} sammen med Kubernetes\footnote{\url{https://kubernetes.io/}}.

\subsection{Axios}
Axios\footnote{\url{https://github.com/axios/axios}} et et JavaScript-bibliotek for en promise-basert\footnote{\url{https://leanpub.com/exploring-es6/}} HTTP-klient som kan kjøres i nettlesere og Node.Js. Biblioteket gjør det enklere å lage og behandle asynkrone forespørsler.

\subsection{Git}
Git er et system for versjonskontroll. Et versjonskontrollsystem blir hovedsaklig brukt under utvikling av software og nettsteder. Det kan også brukes til andre type prosjekter som grafisk design og skriving av dokumenter.

Ved å bruke Git opprettes det historikk over alle endringer, samt en sikkerhetskopi av alle versjoner av filene.

En annen fordel ved å bruke Git er at det blir enklere å samarbeide med andre, uten å måtte tenke på at alle må sitte på nyeste versjon av filene.

KILDE (GIT OG GITHUB): \url{https://journals.plos.org/ploscompbiol/article?id=10.1371/journal.pcbi.1004668}
\url{https://searchitoperations.techtarget.com/definition/GitHub}

\subsection{Github}
GitHub.com er et nettsted for å hoste git repositories, og blir mye brukt for \q{Open Source}-prosjekter. 

Github er et sentralisert punkt for å samle en brukers repositories. Ved siden av å hoste repositories, lar GitHub brukere dele repositories med hverandre, lage informasjonssider om prosjektet og opprette saker (issues).
 
\subsection{Overleaf}
Overleaf\footnote{\url{http://web.simmons.edu/~wilsonjd/LIS488/website/OverleafTutorial.pdf}} er en tjeneste som brukes til å lage, redigere og dele akademiske artikler på nettet ved hjelp av LaTeX. Dette er et nettbasert tekstforfattings-program. 
Overleaf v2\footnote{\url{https://no.overleaf.com/learn/how-to/Working_Offline_in_Overleaf}} tilbyr synkronisering med GitHub og Dropbox.

\subsection{Ottomatik.io}
Ottomatik\footnote{\url{https://ottomatik.io/}} er en webbasert tjeneste for automatisk sikkerhetskopiering av filer og MySQL databaser.

\subsection{Amazon S3}
Amazon Simple Storage Service\footnote{\url{https://aws.amazon.com/s3/}} er en webbasert tjeneste som tilbyr objekt-lagring. Tjenesten tilbyr meget god skalerbarhet, datatilgjengelighet, sikkerhet og ytelse.

KILDE: https://dash.harvard.edu/handle/1/24829568

\subsection{HTTP/2}
HTTP/2\footnote{\url{https://www.rfc-editor.org/rfc/pdfrfc/rfc7540.txt.pdf}} er en revisjon av HTTP-nettverksprotokollen. HTTP er et sett med regler for overføring av filer (tekst, grafiske bilder, lyd, video og andre mediefiler). Et av de store målene med HTTP/2 var å tillate multipleksing.

\subsection{HTTPS}
HTTPS\footnote{\url{https://www.rfc-editor.org/rfc/pdfrfc/rfc2616.txt.pdf}} (Hypertext Transfer Protocol Secure) er en sikker versjon av HTTP-protokollen som overfører data på nett. HTTPS-protokollen krever at det opprettes en kryptert kanal mellom nettleseren og webserveren, slik at overføring mellom disse blir sikker.

\subsection{TLS}
Transport Layer Security er en kryptografisk protokoll. Dette er viktig for å sikre informasjon som overføres gjennom nettverk. TLS brukes i forskjellige tjenester som HTTPS, FTPS, VoIP og VPN.
Når en tjeneste bruker en sikker tilkobling, legges bokstaven S til tjenestens protokollnavn. For eksempel: HTTPS, SMTPS, FTPS, SHIPS.
SSL (Secure Socket Layer) er en utdatert forgjenger til TLS.

KILDE: Thomas, S. (2000). SSL and TLS essentials. New Yourk, 3.

\subsection{Let’s Encrypt}
Let’s Encrypt\footnote{\url{https://letsencrypt.org/}} er en gratis og åpen sertifikatautoritet som gir ut X.509-sertifikater for kryptering av transportlaget. Tjenesten leveres av ISRG (Internet Security Research Group).

\subsection{HTML}
Hypertext Markup Language\footnote{\url{https://www.w3.org/standards/webdesign/htmlcss}} er et markeringsspråk som brukes til å lage nettsidedokumenter. Det er et system som identifiserer og beskriver de forskjellige komponentene i et dokument som overskrifter, avsnitt og lister. HTML5 er den nyeste versjonen av HTML-standarden.

\subsection{CSS}
Casecading Style Sheet\footnote{https://www.w3.org/standards/webdesign/htmlcss} er et format for stilsett som beskriver utseendet i en applikasjon eller på et nettsted. CSS styrer blant annet fonter, farger, bakgrunnsbilder, linjeavstand og sidelayout. Formatet lar utviklere style innhold på en slik måte at brukere enklere kan identifisere hva det er, samt være i tråd med en bedrift eller person sin identitet.

\subsection{SASS}
Syntactically Awesome Style Sheets\footnote{\url{https://sass-lang.com/}} er en utvidelse av CSS. SASS gir mulighet til å bruke variabler, nestede regler, inline-import og mixins. Alt dette gjør det enklere å gjenbruke CSS syntaks. Med SASS er det mulig å lage stilark raskere. SASS er kompatibel med alle versjoner av CSS. Det er ingen nettlesere som kjører SASS, man må derfor kompilere SASS til CSS for å kunne bruke det på nettsider.

KILDE: Cederholm, D. (2013). Sass for web designers. A Book Apart.

\subsection{JavaScript}
JavaScript\url{https://developer.mozilla.org/en-US/docs/Web/JavaScript} er et programmeringsspråk som følger ECMAScript-spesifikasjonen. Det er mye brukt til å utvikle webapplikasjoner. Språket støttes av de fleste moderne nettlesere, og brukes til å manipulere elementene på nettsiden og stilene som er brukt på dem. JavaScript støtter objektorientert- og funksjonell programmering.

\subsection{React}
React\footnote{\url{https://reactjs.org/}} er et fleksibelt JavaScript-bibliotek for å bygge brukergrensesnitt, og brukes som visningslaget for webapplikasjoner som følger MVC. React har åpen kildekode og er utviklet av Facebook.

Alternativ kilde: \url{https://www.fullstackreact.com/assets/media/sGEMe/MNzue/30-days-of-react-ebook-fullstackio.pdf}

\subsection{REST-API}
Et API (programmeringsgrensesnitt) er et sett med funksjoner, prosedyrer, metoder eller klasser som brukes av dataprogrammer for å be om tjenester fra operativsystemet eller programvare som er på datamaskinen. En programmerer kan bruke API-er til å lage applikasjoner.

REST (Representational State Transfer) er en arkitektonisk stil for programvare. Et REST-API er et API som følger REST-stilen og de begrensninger som REST definerer.

KILDE: Masse, M. (2011). REST API Design Rulebook: Designing Consistent RESTful Web Service Interfaces. " O'Reilly Media, Inc.".
side 5 og 6

\subsection{Figma}
Figma\footnote{\url{https://www.figma.com/}} er et webbasert designverktøy som åpner muligheten for å samarbeide i sanntid. Med Figma er det også mulig å designe mockups og prototyper av applikasjoner og nettsider.

ekstra kilde: \url{https://www.xfive.co/blog/figma-best-designer-developer-cooperation/}

\subsection{Google Analytics}
Google Analytics\footnote{\url{https://analytics.google.com/analytics/web/}} er en gratis, webbasert tjeneste som gir statistikk og grunnleggende verktøy for analyse av bruksdata, søkemotoroptimalisering og markedsføring.

\subsection{Google Lighthouse}
Google Lighthouse\footnote{\url{https://developers.google.com/web/tools/lighthouse/}} er et gratis verktøy for å sjekke kvaliteten på nettsider. Det analyserer nettsidens tilgjengelighet, hastighet og SEO, samt om nettsiden følger de beste praksiser både generelt sett og for progressive webapplikasjoner. Lighthouse kan kjøres via Google Chrome DevTools, en nettleserutvidelse i Google Chrome eller via nettsiden \url{https://web.dev/}.

\subsection{WAVE}
Web Accessibility Evaluation Tool\footnote{\url{https://wave.webaim.org/about}} er et verktøy som tester universell utforming hos nettsider og hvorvidt de følger visse punkter i WCAG 2.1-standarden og Section 508. Verktøyet er tilgjenglig som en nettleserutvidelse og som en webbasert tjeneste hos \url{https://wave.webaim.org/}.

\clearpage