\clearpage

\section{Konkurrentanalyse}

Målet med konkurrentanalysen er å finne inspirasjon og kunnskap til nettstedet ved å se på eksisterende løsninger. Hvem er oppdragsgiver sine konkurrenter? Hvordan type løsning har disse valgt? Hvilke styrker og svakheter har løsningene? 

\subsection{Konkurrentene}
Sirkus Media har mange konkurrenter, deriblant alle firmaer som driver mediebyråvirksomhet. Etter samtale med Hans Christian Hymer \footnote{Samtale med oppdragsgiver (16.01.2019)} fikk vi oppgitt 3 bedrifter som blir ansett som Sirkus Media sine største konkurrenter. Det er følgelig disse konkurrentene vi har valgt å analysere.

\subsection{TACTIC™ Real-Time Marketing}
Nettstedet til TACTIC™ Real-Time Marketing ligger på følgende domene; https://tacticrealtime.com

BILDE

\subsubsection{Førsteinntrykk}
Førsteinntrykket er at nettstedet fremstår ryddig og profesjonelt. Det første som møter deg er et stort bakgrunnsbilde. Bilde dekker hele skjermen og har en tittel og en underoverskrift. Headeren består også av en knapp der du kan trykke på "Request a demo". Helt øverst på forsiden er logoen til Tactic og en meny. Til høyre er det en knapp som gir deg muligheten til å se på en video. 

\subsubsection{Forside og undersider}

Forsiden består av bakgrunnsbilde som inneholder logo, meny, overskrifter og en video.  
Deretter kommer det litt tekst om firmaet, informasjon om plattformen, logo av ledende  Ad-servers og DSP`s, partnerprogram og utvalgte kunder. Siden avsluttes med en "Contact us"-oss seksjon.

Nettstedet består også av følgende undersider:
\begin{itemize}
\item Platform
\item Company
\item Partners
\item Clients
\item Blog
\end{itemize}

\subsubsection{Teknisk analyse}
Bruke google lighthouse



\subsubsection{Styrker}
Førsteinntrykket er bra. Nettstedet fremstår som profesjonelt, noe som styrker inntrykket av bedriften. Nettstedet har også god oppbygning og struktur. 

\subsubsection{Svakheter}
Hele nettstedet er på engelsk i tillegg til at det er et .com-domene. Dette kan forvirre brukerne. Vanskelig å skjønne at dette er et firma som er lokalisert i Oslo. En annen svakhet er at det er lite luft rundt noen elementer, som fører til at nettstedet fremstår som kompakt og mer rotete. Det er heller ikke samme mengde luft over og under elementene. Dette er også med på å trekke ned helhetsinntrykket. Det er heller ikke mulig å bevege seg rundt på nettstedet ved hjelp av tastaturet. Dette gjør det også vanskelig å navigere seg rundt ved hjelp av skjermleser, noe som er et krav i loven om universell utforming og tilgjengelighet. I tillegg får nettstedet mange kontrastfeil \footnote{Sjekk av kontraster ble gjort med hjelp av http://wave.webaim.org,}, som også bryter kravet om universell utforming.

\subsection{Marketer Technologies AS}
Nettstedet til Marketer Technologies AS ligger på følgende domene;
https://marketer.tech/

\subsubsection{Førsteinntrykk}
Førsteinntrykket av nettstedet er positivt, fordi den fremstår som ryddig og profesjonell. Det første som møter deg er logoen til firmaet og en meny. Under menyen er en tittel, en underoverskrift og to lenker. Får også opp et ikon for chat. Det at man enkelt kan ta kontakt kan for en kunde virke betryggende, og kan være med å øke troverdigheten til bedriften.

BILDE

\subsubsection{Forside og undersider}

Nettstedet består også av følgende undersider:
\begin{itemize}
\item Solutions
\item Company
\item Customers
\item Blog
\item Careers 
\item Contact
\item Book demo
\end{itemize}

\subsubsection{Teknisk analyse}

\subsubsection{Styrker}
Bra førsteinntrykk. De har chat. Nettsted som skiller seg ut. Gjennomført.

\subsubsection{Svakheter}



\subsection{Inviso AS}
Nettstedet til Inviso AS ligger på følgende domene;
https://www.inviso.no/

\subsubsection{Førsteinntrykk}
Førsteinntrykket av nettstedet er at den fremstår som rotete. Hovedgrunnen til dette er at det er mange elementer og mye som skjer i headeren. Logoen og noe av teksten på bakgrunnsbilde er også vanskelig å se, på grunn av for dårlig kontrast. 

\subsubsection{Forside og undersider}

Nettstedet består også av følgende undersider:
\begin{itemize}
\item Inviso
\item Våre produkter
\item Kontakt
\item Galleri
\item Verktøy
\item Nyheter
\end{itemize}


\subsubsection{Teknisk analyse}

\subsubsection{Styrker}

\subsubsection{Svakheter}

\subsection{Konklusjon}

Burde også presentere sine kunder. Dette er tillitsvekkende og fører til bedre troverdighet.

\clearpage