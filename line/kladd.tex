\clearpage

\section{Konkurrentanalyse}

Målet med konkurrentanalysen er å finne inspirasjon til nettstedet vi skal utvikle ved å se på allerede eksisterende løsninger. Hvem er oppdragsgiver sine konkurrenter? Hvordan type løsning har disse valgt? Hvilke styrker og svakheter har løsningene? Analysen er først og fremst gjort fra en vanlig bruker sitt perspektiv, men noen tekniske faktorer har også blitt vurdert.

\subsection{Teknisk analyse}
For å vurdere de tekniske faktorene har vi brukt Google Lighthouse til å sjekke hvor raskt nettstedene laster inn, beste praksis og SEO. I tillegg ble WAVE \footnote{Web Accessibility Evulation Tool. http://wave.webaim.org/} brukt til å sjekke om nettstedene følger kravene for loven om universell utforming og tilgjengelighet.

\subsection{Konkurrentene}
Sirkus Media har mange konkurrenter, deriblant alle firmaer som driver mediebyråvirksomhet. Etter samtale med Hans Christian Hymer \footnote{Samtale med oppdragsgiver (16.01.2019)} fikk vi oppgitt 3 bedrifter som blir ansett som Sirkus Media sine største konkurrenter. Det er følgelig disse konkurrentene vi har valgt å analysere.

\subsection{TACTIC™ Real-Time Marketing}
Nettstedet til TACTIC™ Real-Time Marketing ligger på følgende domene; https://tacticrealtime.com

\subsubsection{Førsteinntrykk}
Førsteinntrykket er at nettstedet fremstår ryddig og profesjonelt. Det første som møter deg er et stort bakgrunnsbilde. Bilde dekker hele skjermen og har en tittel og en underoverskrift. Headeren består også av en knapp der du kan trykke på \q{Request a demo}. Helt øverst på forsiden er logoen til Tactic og en meny. Til høyre er det en knapp som gir deg muligheten til å se på en video. 

\begin{figure}[H]
    \centering
    \includegraphics[width=\textwidth]{line/tacticrealtime_com_(1366x768).png}
    \caption{tacticrealtime.com}
    \label{fig:competitors-tacticrealtime.com}
\end{figure}

\subsubsection{Forside og undersider}

Forsiden består av bakgrunnsbilde som inneholder logo, meny, overskrifter og en video.  
Deretter kommer det litt tekst om firmaet, informasjon om plattformen, logo av ledende  Ad-servers og DSP's, partnerprogram og utvalgte kunder. Siden avsluttes med en \q{Contact us}-seksjon.

Nettstedet består også av følgende undersider:
\begin{itemize}
\item Platform
\item Company
\item Partners
\item Clients
\item Blog
\end{itemize}

\subsubsection{Teknisk analyse}

\begin{figure}[H]
    \begin{center}
        \subfigure[Google lighthouse]{
            \label{fig:competitors-lighthouse-summary-tacticrealtime.com}
            \includegraphics[width=0.7\textwidth]{line/tacticrealtime_com-lighthouse.png}
        }
        \rulesepv
        \subfigure[WAVE]{
            \label{fig:competitors-wave-summary-tacticrealtime.com}
            \includegraphics[width=0.2\textwidth]{line/tacticrealtime_com-wave.png}
        }
        
        \label{fig:competitors-tech_analysis-tacticrealtime.com}
        \caption{Resultat for teknisk analyse av tacticrealtime.com}
    \end{center}
\end{figure}



% KLADDIS - FJERN
Figur \ref{fig:competitors-lighthouse-summary-tacticrealtime.com} viser test resultater fra Google Lighthouse. Se vedlegg X for full rapport på hver av disse. Merk at PWA ikke viser resultat som den skal, men det gjør ingenting ettersom det ikke er så relevant for disse nettstedene uansett \footnote{Se avsnitt \ref{sec:analysis-current-lighthouse} for en utdypende forklaring}.
Vi har satt oss som mål å få minst like bra score som gjennomsnittet til disse.


\subsubsection{Styrker}
Førsteinntrykket er bra. Nettstedet fremstår som profesjonelt, noe som styrker inntrykket av bedriften. Nettstedet har også god oppbygning og struktur. 
Sterk profil. Det er ganske tydelig at grønn er pimærfargen til selskapet.

\subsubsection{Svakheter}
Hele nettstedet er på engelsk i tillegg til at det er et .com-domene. Dette kan forvirre brukerne. Vanskelig å skjønne at dette er et firma som er lokalisert i Oslo. En annen svakhet er at det er lite luft rundt noen elementer, som fører til at nettstedet fremstår som kompakt og mer rotete. Det er heller ikke samme mengde luft over og under elementene. Dette er også med på å trekke ned helhetsinntrykket. Det er heller ikke mulig å bevege seg rundt på nettstedet ved hjelp av tastaturet. Dette gjør det også vanskelig å navigere seg rundt ved hjelp av skjermleser, noe som er et krav i loven om universell utforming og tilgjengelighet. I tillegg får nettstedet mange kontrastfeil, som også bryter lovkravet om universell utforming.

\subsection{Marketer Technologies AS}
Nettstedet til Marketer Technologies AS ligger på følgende domene;
https://marketer.tech/

\subsubsection{Førsteinntrykk}
Førsteinntrykket av nettstedet er positivt, fordi den fremstår som ryddig og profesjonell. Det første som møter deg er headeren til nettstedet, som inneholder logoen til firmaet og en meny. Under menyen er en tittel, en underoverskrift og to lenker. Får også opp et ikon for chat. Det at man enkelt kan ta kontakt kan for en kunde virke betryggende, og kan være med å øke troverdigheten til bedriften.

\begin{figure}[H]
    \centering
    \includegraphics[width=\textwidth]{line/marketer_tech_(1366x768).png}
    \caption{marketer.tech}
    \label{fig:competitors-marketer.tech}
\end{figure}

\subsubsection{Forside og undersider}

Forsiden, i tillegg til headeren, består av logo til noen utvalgre kunder, prosessen til Marketer, animasjoner, resultater fra tidligere prosjekter og en presentasjon av deres løsninger. Helt nederst på siden kommer bunnfeltet, der organisasjonsnummeret, kontaktinformasjon og adresse blir presentert.

Nettstedet består også av følgende undersider:
\begin{itemize}
\item Solutions
\item Company
\item Customers
\item Blog
\item Careers 
\item Contact
\item Book demo
\end{itemize}

\subsubsection{Teknisk analyse}
5 kontrastfeil. Hele 147 feil når det kommer til universell utforming og tilgjengelighet. 135 av disse feilene er manglende bruk av alt-tekster.

\begin{figure}[H]
    \begin{center}
        \subfigure[Google lighthouse]{
            \label{fig:competitors-lighthouse-summary-marketer.tech}
            \includegraphics[width=0.7\textwidth]{line/marketer_tech-lighthouse.png}
        }
        \rulesepv
        \subfigure[WAVE]{
            \label{fig:competitors-wave-summary-marketer.tech}
            \includegraphics[width=0.2\textwidth]{line/marketer_tech-wave.png}
        }
        
        \label{fig:competitors-tech_analysis-marketer.tech}
        \caption{Resultat for teknisk analyse av marketer.tech}
    \end{center}
\end{figure}

\subsubsection{Styrker}
En av styrkene er at nettstedet gir et bra førsteinntrykk. Det er også et nettsted som skiller seg ut og er gjennomført. En annen styrke er at løsningen inneholder chat, der kunder enkelt kan ta kontakt. Profilen er også klar og tydelig, og det er åpenbart at bedriften har sort som primærfarge og turkis som aksentfarge.

\subsubsection{Svakheter}
En ulempe er at nettstedet består av flere store filer, som fører til at nettstedet oppleves som tregt. En annen svakhet er at løsningen får mange feil når det kjøres tester på universell utforming og tilgjengelighet.


\subsection{Inviso AS}
Nettstedet til Inviso AS ligger på følgende domene;
https://www.inviso.no/

\subsubsection{Førsteinntrykk}
Førsteinntrykket av nettstedet er at den fremstår som rotete. Hovedgrunnen til dette er at det er mange elementer og mye som skjer i headeren\footnote{Hodet til nettstedet}. Logoen og noe av teksten på bakgrunnsbilde er også vanskelig å se, på grunn av for dårlig kontrast. 

\begin{figure}[H]
    \centering
    \includegraphics[width=\textwidth]{line/inviso_no_(1366x768).png}
    \caption{inviso.no}
    \label{fig:competitors-inviso.no}
\end{figure}

\subsubsection{Forside og undersider}
I tillegg til headeren består forsiden av \q{Om oss}-tekst, deres tjenester og utvalgte kunder. 

Nettstedet består også av følgende undersider:
\begin{itemize}
\item Inviso
\item Våre produkter
\item Kontakt
\item Galleri
\item Verktøy
\item Nyheter
\end{itemize}


\subsubsection{Teknisk analyse}
Ved å kjøre test via WAVE oppgir resultatet at nettstedet har 4 kontrastfeil. I tillegg eksisterer det 17 bilder som mangler alt-tekst. 

\begin{figure}[H]
    \begin{center}
        \subfigure[Google lighthouse]{
            \label{fig:competitors-lighthouse-summary-inviso.no}
            \includegraphics[width=0.7\textwidth]{line/inviso_no-lighthouse.png}
        }
        \rulesepv
        \subfigure[WAVE]{
            \label{fig:competitors-wave-summary-inviso.no}
            \includegraphics[width=0.2\textwidth]{line/inviso_no-wave.png}
        }
        
        \label{fig:competitors-tech_analysis-inviso.no}
        \caption{Resultat for teknisk analyse av inviso.no}
    \end{center}
\end{figure}

\subsubsection{Styrker} 
Bra profil. Det er lett å skjønne at rød er primærfargen til Inviso. Det tekstlige innholdet på siden er godt skrevet og får bedriften til å fremstå profesjonelt.

\subsubsection{Svakheter}
En svakhet er at det er for mye informasjon i headeren. Dette kan bidra til at førsteintrykket blir dratt ned. En annen stor svakhet er at nettstedet får mange feil når man kjører tester når det gjelder universell utforming og tilgjengelighet. Det faktum at nettsiden har en mobil-meny på fullversjon av siden, trekker også ned fordi dette gjør brukervennligheten dårligere.

\subsection{Konklusjon}
Felles for alle 3 konkurrentene er at ingen av de følger loven om universell utforming og tilgjengelighet. Universell utforming er lovpålagt i Norge \footnote{\url{https://lovdata.no/dokument/SF/forskrift/2013-06-21-732}}, og vår løsning kommer derfor til å følge disse kravene. 

En annen egenskap alle har til felles er at de presenterer sine kunder og/eller partnere. Dette bør også Sirkus Media gjøre. Presentasjon av kunder som er kjente firmaer er tillitsvekkende og kan skape bedre troverdighet blant brukerne.

Figur \ref{fig:competitors-mobile} viser at alle konkurrentene har gått for samme løsning når det kommer til design på mobil. Alle 3 har bedriftens logo oppe til venstre og et meny ikon til høyre.
Planen er å at den nye nettsiden skal ha en annen løsning.  Undersøkelser \footnote{Kilde: Luke Wroblewski, en internasjonalt anerkjent brukeropplevelsesdesigner \url{https://www.lukew.com/ff/entry.asp?1945}} viser at dette er langt fra den mest optimale løsningen.

Konkurrentene har gode løsninger, men også en del svakheter som har blitt påpekt i denne analysen. Studentgruppen har etter analysen fått inspirasjon på til deler av løsningen, men også en pekepinn på hva som bør gjøres annerledes.

\begin{figure}[bh]
    \begin{center}
        \includegraphics[width=0.3\textwidth]{line/tacticrealtime_com_(iPhone_6_7_8).png}
        \includegraphics[width=0.3\textwidth]{line/marketer_tech_(iPhone_6_7_8).png}
        \includegraphics[width=0.3\textwidth]{line/inviso_no_(iPhone_6_7_8).png}
        \caption{Alle nettstedene er mobiltilpasset}
        \label{fig:competitors-mobile}
    \end{center}
\end{figure}

\clearpage