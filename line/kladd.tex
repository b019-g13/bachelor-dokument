\clearpage

\section{Planlegging av struktur og innhold}

Etter analysen gjort i kapittel \ref{chap:analysis} har studentgruppen kommet frem til at struktur og innhold burde være som følgende: 

\textbf{Forside} Tanken er at hele nettstedet i hovedsak skal bestå av en lang forside. Fordi dette ikke er beste praksis for søkemotoroptimalisering vil vi forbedre SEO ved å lenke til nye sider som er mer detaljert. Høyt oppe på siden skal det være et kontaktskjema. Telefonnummeret til bedriften skal også være lett tilgjengelig. I tillegg skal forsiden bestå av en beskrivelse av bedriften, presentasjon av hva de tilbyr, steg for steg om prosessessen med animasjoner og omtale og logo fra eksisterende kunder. Videre skal nettstedet også ha et kart som er integrert med Google Maps, som viser lokasjonen til bedriften. Nederst på nettstedet skal det være organisasjonsnummer, link til personvern og logg inn.

\textbf{Meny} Nettstedets meny skal være tilgjengelig uavhengig av hvilken side brukeren står på. Her skal alltid forside, om oss og kontakt oss være tilgjengelig. 

\textbf{Om oss} Det vil også bli opprettet en egen \q{Om oss}-side som man kommer til ved å trykke \q{Les mer} på forsiden. Denne siden inneholder en mer detaljert presentasjon av Sirkus Media og deres historie.

\textbf{Kontakt oss} Nettstedet vil i tillegg til kontaktskjema på forsiden, bestå av en egen side med kontaktinformasjon. Her presenteres også relevant informasjon som bedriftens adresse, e-post, mobilnummer og kart. Vi lager en dedikert side for kontaktinformasjon fordi det er positivt med tanke på søkemotoroptimalisering. For eksempel: Hvis en bruker søker på Google etter Sirkus Media og det ikke er en egen side for kontakt oss, vil brukeren kun få treff på forsiden og undersidene med andre urelevante temaer. Siden brukeren ønsker å kontakt Sirkus Media,  hadde det vært mer brukervennlig å ha en link til kontakt siden i Google.

\textbf{Logg inn} Egen side der de ansatte kan logge inn for å komme til brukergrensesnittet for oppdatering og endring av innhold.

\textbf{Administrasjonsside}  På administrasjonssiden kan de ansatte endre og legge til innhold gjennom et brukergrensesnitt.

\textbf{Informasjon om personvern og informasjonskapsler} Personvernssiden skal inneholde en personvernserklæring. Denner forteller hvordan Sirkus Media samler inn og bruker personopplysninger. Målet er å gi brukerne overordnet informasjon om deres behandling av personopplysninger. I tillegg skal den fortelle brukeren hvilke informasjonskapsler som blir brukt på nettstedet.

\textbf{Annet} Alle sidene skal ha tilgang til en chat, der man kan ta direkte kontakt med oppdragsgiver.

\section{Definering av CMS}

\section{Planlegging av database}

På forhånd var det visse faktorer vi visste vi måtte ta hensyn til:


\section{Front-end}

\subsection{Oppsett av React og tilhørende verktøy}

Lastet først ned Node.js. Blir ikke brukt som en back-end tjenester eller server, blir brukt til å kunne kjøre alle verktøy som må til for å kunne utvikle lokalt på datamaskinen. https://nodejs.org/en/

Lastet ned nyeste versjonen (11.9.0)

Deretter ble nettleserutvidelsen React Developer Tools lagt til fra Google Chrome sin webstore.

I tillegg vil man trenge et kommandolinjevindu og en teksteditor. Jeg bruker kommandolinjevindu Powershell og teksteditoren Sublime.

Opprettet et repository på Github og lastet ned på maskinen ved hjelp av GitKraken. 

Fulgte guiden til react.org for å sette opp en ny React App. https://reactjs.org/docs/create-a-new-react-app.html#create-react-app



Kunne ikke kjøre create react app. Fikk feilmeldinger
Fant noen som hadde samme problem: https://github.com/facebook/react/issues/11933
Måtte installere yarn og create-react-app globalt med yarn.
La så til C:\Users\Datahjelpen AS\AppData\Local\Yarn\bin i windows PATH så jeg kunne kjøre create-react-app
Da fungerte alt.


\clearpage