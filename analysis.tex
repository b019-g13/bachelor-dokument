\cleardoublepage
\chapter{Analyse og teori}
\label{chap:analysis}
\meta{
Kapittelet tar for seg analysedelen av arbeidet. Den består av to hoveddeler, en grundig beskrivelse av oppgaven basert på skissen gitt av oppdragsgiver, og en undersøkelse av hva som finnes av relatert arbeid, {\em best practise} og relevant teknologi. 
}

I dette kapittelet danner vi oss et bilde av hvordan en akademisk/teknisk rapport bør se ut, både med hensyn på innhold, struktur og utforming. Vi ser også på hvordan store og komplekse dokumenter blir produsert (både analogt og digitalt), og hva slags (digitale) verktøy som kan tenkes å brukes. Vi går også gjennom endel tilgjengelig materiale som kan gjøre det lettere å velge hva slags digitale verktøy man burde bruke i en bacheloroppgave\footnote{Utvalget er noe overfladisk utført, i hovedsak basert på forfatterens erfaringer.}.

-- 

Kapittel 2 omhandler relevant teori og beste praksis for planlegging, design og utvikling. I dette kapittelet vil vi danne oss et bilde av hvordan en nettsted bør se ut. Både med hensyn på innhold, struktur og utforming. I tillegg vil vi presentere de verktøy vi mener er passende i et slikt prosjekt.

\section{Utdypning av oppgavebeskrivelse}
Sirkus Media har gitt utrykk for at de ønsker en "one-pager". Dette vil si at nettstedet i hovedsak består av en langt nettside. Samtidig ønsker oppdragsgiver at nettstedet skal være godt synlig i Google og andre søkemotorer og at det finnes mulighet for innlogging slik at det er mulig å legge til, oppdatere og fjerne innhold. Studengruppen skal dermed utvikle et eget CMS eller benytte et ferdigutviklet system som møter disse behovene. 

CMS og API

\section{Hvorfor et nettsted er viktig}
Å benytte seg av internett har i dagens samfunn blitt et hverdagslig gjøremål for mange. I følge undersøkelser gjort av Statistisk sentralbyrå (SSB)\footnote{Statistisk sentralbyrå utarbeider og gir ut offisiell statistikk i Norge} i 2017, hadde hele 98\% av Norges befolkning tilgang på internett hjemme \cite{ssb17fim}. Det viste seg også at 9 av 10 nordmenn surfet på nettet hver dag \cite{ssb17nat}. SSB har også kartlagt  at så mange som 89\% av Norges befolkning, mellom 16-79 år, brukte internett til å søke etter informasjon om varer eller tjenester de siste 3 måneder i 2018 \cite{ssb18aup}. Disse foregående funnene er svært relevante, og er med å underbygge hvorfor det er så viktig for en bedrift å være på internett i dag.

I tillegg til disse funnene, er det også andre faktorer som bidrar til at et nettsted for en bedrift er så viktig i dag. Disse blir presentert nærmere i dette kapittelet. 

\subsection{Treff i søk}
Når en bedrift har et nettsted blir det mulig å dukke opp i søkeresultatene hos søkemotorer\footnote{Se avsnitt \ref{sec:concepts-seo}}. I følge undersøkelser gjort av Google viser det seg at hele 76\% av de som søker på smarttelefonene etter noe i nærheten, besøker en bedrift innen en dag \cite{google16hms}. I tillegg viser undersøkelsene at når et spørsmål eller behov oppstår er sannsynligheten minst dobbelt så stor for å benytte seg av søk, enn andre online og offline kilder som sosiale medier eller å besøke butikker. Ikke bare er søk den mest brukte ressursen, det er den tjenesten 87\% vender seg til først \cite{google16mhc}. I undersøkelsene ble det også kartlagt at 51\% av brukere har oppdaget et nytt produkt eller bedrift når de har utført søk på deres telefon \cite{google16scp}. For bedrifter er det altså helt avgjørende å være godt synlig hos søkemotorer.




\subsection{Troverdighet}
Etter studier gjort av avdelingen for informasjon og ledelse ved HuaZhong Normal University viser det seg at troverdihgeten til en bedrifts nettsted er en viktig faktor når brukerene skal bestemme seg for om prosessen skal gå videre eller ikke, og vil ha en stor påvirkning på deres avgjørelse når det gjelder kjøp.


\meta{The level of enterprise website credibility is an important basis for the
website browsers (or consumers) deciding whether to take
further action or not, and will have a great influence on their
purchase decision-making. }

I følge studiene kommer det også frem at kriteriene for å bestemme troverdigheten til et nettsted er basert på subjektiv vurdering og ikke profesjonell analyse. Brukere undersøker vanligvis utseende, informasjonen og funksjonaliteten til nettstedet i henhold til deres faktiske behov og gjør seg deretter opp deres egen konklusjon. Vi kan klassifisere disse faktorene, i forhold til å dømme troverdigheten, i tre typer: 

\meta{As the above study shows, criterion for determining the
credibility of website is based on subjective judgment instead
of professional analysis. People usually examine the
appearance, information and function of the website
according to their actual needs and then get their own
conclusion. We can classify these factors, in terms with
judging the credibility, into three types: website image,
information content and business function. Accordingly,
from the online consumers’ perspective, the enterprise
website credibility can be divided into three parts: the
website image credibility, the information content credibility
and the business function credibility}
\url{https://ieeexplore.ieee.org/stamp/stamp.jsp?tp=&arnumber=5279898}

\cite{zhao2009eew}

I en studie gjennomført av  "brightlocal" i 2016 ble det gjort undersøkelser på hva forbrukere ønsker fra lokale bedrifter. Resultatene ble tatt fra et panel på 800 forbrukere, der alle er basert i USA, med forskjellig alder og kjønn. Blant de 800 respondentene, var alderspennet følgende: 

\begin{itemize}
\item 18-34 – 44\%
\item 35-54 – 34\%
\item 55+ – 22\%
\end{itemize}

I denne undersøkelsen svarte 40\% av aldergruppen mellom 18-34 og 35-54 at det er mer sannsynlig at de kontakter en lokal bedrift hvis de har en nettside. På spørsmål om en klar og smart nettside øker bedriftens troverdighet, svarte 43\% av respondendene over 55 at dette hadde noe å si \cite{marchant18wdc}.

Med utgangspunktet i disse studiene kan vi derfor konkludere med at en nettside med god troverdighet kan føre til at flere potensielle kunder tar kontakt.

\subsection{Markedsføringskanal}
Nettsteder kan ses på som et eksempel på et enveis nettverktøy med høy  bedrifts kontroll.
Websites and e-mail can be seen as examples of one-way online tools with high company control. https://www.emeraldinsight.com/doi/pdfplus/10.1108/JSBED-05-2013-0073. På en nettside har en bedrift altså mulighet til å bestemme innholdet og hvordan firmaet skal fremstå. 

\subsection{Innsamling av data}

\section{Strategi og planlegging}
I alle type prosjekter er det viktig å ha en strategi og planlegge etter dette. Kilde. Derfor kan det være helt avgjørende å ha en god plan.

Når det kommer til webdesign-prosjekter inneholder disse fire viktige komponenter. Første steg er at du må kjenne miljøet ditt. Andre steg er å planlegge. Tredje steg er å lære å kunne tilpasse seg, og siste steg er å fullføre prosessen ved å løse kompatibilitetsproblemer etter hvert som de oppstår. (Dawson, 2011).

Dawson, A. (2011). Future-proof web design. Retrieved from https://ebookcentral.proquest.com

\url{https://www.sciencedirect.com/science/article/pii/S014829631630203X#bb0175 }

\section{Relevante begreper og viktigheten av disse}

\subsection{Personas og målgruppe}

Personas er fiktive og detaljerte beskrivelser av kunder som fungerer som modeller eller test-caser for produktdesign. Personas er mye brukt i nettsidedesign , samt for markedsføringsrelatert behovsanalyse \cite[s.~130]{grenci2007mcv}.

\url{https://www.thinkwithgoogle.com/marketing-resources/tutorials/human-insights-audience-strategy/}

En målgruppe

Personas vs målgruppe:


\subsection{Design, profil og førsteinntrykk}
\url{https://www.tandfonline.com/doi/pdf/10.1080/10447318.2013.839899?needAccess=true }

\url{http://news.mst.edu/2012/02/eye-tracking_studies_show_firs/}

\url{https://www.tandfonline.com/doi/abs/10.1080/01449290500330448 }


\subsection{Responsivt design}
Undersøkelser fra Google om mobilbruk... Står om det i en av kildene som har blitt brukt tidligere

\subsection{Struktur og oppbygning}

\subsection{Søkemotoroptimalisering}
\label{sec:concepts-seo}
Søkemotoroptimalisering (SEO) går ut på å oppnå gode plasseringer i de organiske søkeresulatene hos søkemotorer \cite[s.~16]{flensted10smg}. I praksis handler SEO om at en nettside blir optimalisert på en rekke områder, slik at søkemotoren finner akkurat dette nettstedet og presenterer den godt synlig i de organiske søkeresultatene ved å se på søkeordene ønsket målgruppe benytter  \cite[s.~20]{flensted10smg}.

Mer teoretisk fungerer søkemotoroptimalisering ved at de innsamlede dataene blir indeksert og lagret i en database. Alle disse operasjonene utføres av søkemotorprogramvaren (crawler, edderkopp,
bot). Disse programmene beveger seg ved å bruke hyperkobling struktur på nettet. De navigerer regelmessig gjennom nettsider og fanger endringer som har blitt gjort siden sist  \cite[s.~488]{yalccin2010search}.

God SEO krever disiplin, planlegging og research før man setter i gang. Dette er ikke en oppgave som kun gjøres en gang, og krever konstant overvåking og optimalisering da algoritmene hele tiden endrer hvordan de evaluerer nettsider \cite[s.~1]{mitchell2012usb}.

Undersøkelser viser at 75\% of Internet users never scroll past the first page of search results, TRENGER KILDE!! 
Disse funnene underbygger påstanden om at SEO er viktig i dagens samfunn, og en god SEO-strategi er derfor på sin plass. 
\url{https://www.searchenginepeople.com/blog/40-unbelievable-seo-statistics-need-know.html#note-68961-5 }

\subsection{Brukeropplevelse}
I følge Steve Krug, forfatteren bak boken \q{Don't make me think}, går brukervennlighet ut på noe så enkelt som at en helt vanlig bruker med gjennomsnittlige egenskaper og erfaring (eller til og med lavere), skal kunne finne ut av hvordan en ting skal brukes for å oppnå et mål, uten å dette fører til mer trøbbel enn det er verdt \cite[s.~9]{krug2014dmt}. Det er nettopp derfor forfatterens første lov til god brukeropplevelse er \q{ik ke få meg til å tenke}. Et nettsted skal nemlig være selvforklarende, og en bruker skal intuitivt forstå hva et nettsted representerer og hvordan den kan brukes, uten å måtte anstrenge seg for å finne ut av det \cite[s.~11]{krug2014dmt}. 

\url{https://www.thinkwithgoogle.com/marketing-resources/experience-design/strong-audience-design/}

\subsubsection{Hurtighet}
\url{https://www.section.io/blog/page-load-time-bounce-rate/ }


\section{Lovverk}
\subsection{WCAG 2.0}
\subsection{GDPR}



\section{Beste praksis og tidligere undersøkelser}
Undersøkelser viser at følgende struktur på en nettside er å foretrekke...:

Finne et nettsted som følger best practice og skrive om det.

Obvious Always Wins:
\url{https://www.lukew.com/ff/entry.asp?1945}

\subsection{Innhold}
https://www.brightlocal.com/2016/03/04/gender-age-what-consumers-want-from-local-business-websites/ 
\subsection{Struktur}

\subsection{Sikkerhet}
 (hashing av passord. sende data over kryptert kanal)

\section{Relatert arbeid}

Finne nettsider som har vunnet priser.


\section{Teknologi}
Når det kommer til utvikling av nettsteder finnes det mange ulike teknologier, systemer og verktøy som kan benyttes for å få det samme resultatet. Derfor er det viktig å kartlegge brukerens behov og deretter sette ønskede kvalifikasjoner og krav, slik at de valgte verktøyene er tilpasset dette. Basert på foregående analyse i dette kapittelet har studentgruppen foreslått et sett med minimumskrav. Videre blir passende verktøy og tilhørende teknologier presentert.

\subsection{Krav til teknologier}
Følgende sett med minimumskrav er satt med basis i den foregående analysen:


\subsection{Designmønster}
\subsection{Verktøy til grafisk design}
\subsection{Verktøy til back-end}
\subsection{Verktøy til front-end}
\subsection{Andre verktøy}
\subsection{Tilhørende teknologier og begreper}

\subsubsection{Laravel}
Laravel\footnote{\url{https://laravel.com/}} er et PHP-rammeverk, med åpen kildekode, som brukes til å lage webapplikasjoner og andre nettsteder. Rammeverket baserer seg på Symfony\footnote{Avsnitt \ref{sec:tools-symfony}} og følger Model-view-controller\footnote{Avsnitt \ref{sec:tools-mvc}} (MVC) designmønsteret.
Laravel kommer med innebygde funksjoner for autentisering, modeller, visninger, sesjoner, ruting og andre funksjoner som en \textit{templating engine} kalt Blade\footnote{\url{https://laravel.com/docs/5.7/blade}}.

%%% VI SER PÅ FORDELER OG ULEMPER SENERE %%%
%\textbf{Fordeler}
%\begin{itemize}
%    \item Laravel er veldig raskt i forhold til CMS (Content Management Systems).
%    \item Har funksjoner som autentisering og autorisasjon.
%    \item Alt kan tilpasses og styres hvordan tingene fungerer.
%    \item Databasen kan brukes eller utformes på egen måte.
%\end{itemize}

%\textbf{Ulemper}
%\begin{itemize}
%    \item I Laravel må SEO definere egne ruter, og det tar mye arbeid å utvikle et nettsted som hovedsakelig er avhengig av innhold.
%    \item Laravel rammeverk er litt komplisert og krever mer kunnskap.
%    \item Laravel er mindre fleksibel for å oppdatere innholdet.
%\end{itemize}

%Fordeler og Ulemper av laravel ble tatt i forhold til CMS.
%Kilde: \url{https://www.educba.com/laravel-vs-wordpress/}

\subsubsection{Symfony}
\label{sec:tools-symfony}
Symfony er et webapplikasjonsrammeverk og er et sett med gjenbrukbare komponenter og biblioteker skrevet i PHP. \cite{symfony19wis}

\subsubsection{MVC}
\label{sec:tools-mvc}
Model-View-Controller (MVC)\footnote{\url{http://www.dgp.toronto.edu/~dwigdor/teaching/csc2524/2012_F/papers/mvc.pdf}} er et designmønster som deler et program inn i tre hovedkomponenter: modell, visning og kontroll.\cite{burbeck87aps} Oppbygningen kan enkelt forklares ved hjelp av et eksempel fra Matteo publisert på dev.to:
Matteo J. (2018, Mars 16) Re: Explain MVC like I'm five [Blogg kommentar]\footnote{\url{https://dev.to/matteojoliveau/comment/2ikd}}
\begin{quote}
    You walk in a McDonald and you order at the big interactive touchscreen.
    The screen is the View, and you ask for your meal there.
    The order is passed to the operator, who is the Controller, that will go in the kitchen and fetch a burger (your wanted resource) of the type (Model) you asked, then deliver it back to you.
\end{quote}
%Typisk flyt er at det kommer inn en forespørsel som går til kontrolleren. Den tolker forespørselen og behandler den ved å hente og endre data via modell. Til slutt sendes det et svar til visning.

\textbf{Modell} (Model) representerer de forskjellige type ressurser en applikasjon har. Her beskrives hver ressurs, hvordan den er oppbygd og hva som kan gjøres med den.

\textbf{Visning} (View) håndterer brukerinteraksjon og presentasjon av data. Er ofte et grafisk brukergrensesnitt.

\textbf{Kontrolleren} (Control) delen er ansvarlig for å koordinere forespørsler og svar.

\subsubsection{GNU/Linux}
GNU/Linux er en familie med Unix-lignende operativsystemer som baserer seg på Linux-kjernen og en del programvare fra GNU-prosjektet.
I denne familien finner vi blant annet Ubuntu, Debian, Fedora, Red Hat Linux og Arch Linux.
Kilder: \url{https://www.kernel.org} \url{https://www.gnu.org/} \url{https://www.ubuntu.com/} \url{https://www.debian.org/} \url{https://getfedora.org/} \url{https://www.redhat.com/en/topics/linux} \url{https://www.archlinux.org/}

\subsubsection{Nginx}
Nginx er en gratis HTTP-server med åpen kildekode. Nginx har mulighet til å håndtere høy belastning av HTTP-forespørsler. Programmet er tilgjengelig på operativsystemer som Windows, Mac OS og Solaris. I tillegg er Nginx tilgjengelig på operativsystemer som er basert på GNU/Linux eller BSD.

Kilde: Nedelcu, C. (2010). Nginx HTTP Server: Adopt Nginx for Your Web Applications to Make the Most of Your Infrastructure and Serve Pages Faster Than Ever. Packt Publishing Ltd.

\subsubsection{PHP}
PHP (rekursiv akronym for PHP Hypertext Pre-processor) er et skriptspråk som er spesielt egnet for webutvikling og kan legges inn i filer sammen med HTML. PHP er et serverside programmeringsspråk og kan brukes til å lage dynamiske og interaktive nettsteder. PHP er gratis og plattformuavhengig, samtidig som det er  er raskt og fleksibelt. Det kan installeres i pakker med webserver og database. Eksempelvis LAMP, LEMP, WAMP og XAMPP. PHP er også mulig å installere enkeltstående. 

Kilde: \url{http://php.net/}

\subsubsection{Sesjoner}
En sesjon\footnote{\url{https://ieeexplore.ieee.org/abstract/document/8392612}} (session) er en samling av data lagret på en webserver. En webserver tildeler en ID til hver bruker som sender forespørsler via en nettleser. ID-en lagres som en informasjonskapsel i nettlesern til brukeren. Alle nye forespørsel vil så identifiseres av ID-en som er lagret hos brukeren.

\subsubsection{phpMyAdmin}
phpMyAdmin\footnote{\url{https://www.phpmyadmin.net/}} er et gratis og webbasert administrasjonsverktøy for MySQL og MariaDB databaser. Verktøyet er skrevet i PHP og JavaScript. Med phpMyAdmin kan man lage, endre og slette databaser, tabeller og felter. Annen funksjonalitet kan være å utføre SQL-setninger, administrere nøkler og privilegier og eksportere data til ulike formater.

\subsubsection{MariaDB}
MariaDB er et relasjonsdatabasesystem basert på MySQL, som er gratis og har åpen kildekode. Systemet funker som en \textit{drop-in} erstatning for MySQL.

Kilde \url{https://mariadb.org/}

\subsubsection{CSRF}
Cross site request forgery\footnote{\url{https://ieeexplore.ieee.org/abstract/document/5283085}} (CSRF) er angrep som tvinger brukere til å utføre uønskede handlinger på en nettside der brukeren er logget inn, via en annen nettside. For eksempel: hvis brukeren er logget inn på Facebook.com, og går til Twitter.com. Da kan Twitter laste inn et bilde som dette: \lstinline{<img src="https://facebook.com/delete-my-account">}. Dette vil da sende en forespørsel til den linken, som kan føre til sletting av konto. Om Facebook ikke har implementert mottiltak for slike angrep, vil denne forespørselen tolkes som om den kommer direkte fra brukeren.

\subsubsection{CI \& CD}
CI står for continuous integration og CD står for continuous delivery. CI/CD\footnote{\url{https://www.atlassian.com/continuous-delivery/principles/continuous-integration-vs-delivery-vs-deployment}} er en metode som lar utviklere implementere og levere kodeendringer raskt og pålitelig.

\subsubsection{Buddy}
Buddy\footnote{\url{https://buddy.works}} er en webbasert applikasjon for CI og CD. Applikasjonen lar utviklere bygge, teste og distribuere nettsider og programkode automatisk når kildekoden oppdateres. For eksempel er det mulig å koble Buddy til et nettsteds kildekode via Git. Når koden til nettstedet oppdateres, vil Buddy automatisk detektere endringer og starte en byggeprosess. Når byggeprosessen er ferdig kjøres det automatisk tester på koden, som verifiserer at alt fungerer. Deretter vil Buddy laste koden opp til en server og oppdatere nettstedet som ligger ute på nett. Dette skjer gjerne gjennom Docker\footnote{\url{https://www.docker.com/}} sammen med Kubernetes\footnote{\url{https://kubernetes.io/}}.

\subsubsection{Axios}
Axios\footnote{\url{https://github.com/axios/axios}} et et JavaScript-bibliotek for en promise-basert\footnote{\url{https://leanpub.com/exploring-es6/}} HTTP-klient som kan kjøres i nettlesere og Node.Js. Biblioteket gjør det enklere å lage og behandle asynkrone forespørsler.

\subsubsection{Git}
Git er et system for versjonskontroll. Et versjonskontrollsystem blir hovedsaklig brukt under utvikling av software og nettsteder. Det kan også brukes til andre type prosjekter som grafisk design og skriving av dokumenter.

Ved å bruke Git opprettes det historikk over alle endringer, samt en sikkerhetskopi av alle versjoner av filene.

En annen fordel ved å bruke Git er at det blir enklere å samarbeide med andre, uten å måtte tenke på at alle må sitte på nyeste versjon av filene.

KILDE (GIT OG GITHUB): \url{https://journals.plos.org/ploscompbiol/article?id=10.1371/journal.pcbi.1004668}
\url{https://searchitoperations.techtarget.com/definition/GitHub}

\subsubsection{Github}
GitHub.com er et nettsted for å hoste git repositories, og blir mye brukt for \q{Open Source}-prosjekter. 

Github er et sentralisert punkt for å samle en brukers repositories. Ved siden av å hoste repositories, lar GitHub brukere dele repositories med hverandre, lage informasjonssider om prosjektet og opprette saker (issues).
 
\subsubsection{Overleaf}
Overleaf\footnote{\url{http://web.simmons.edu/~wilsonjd/LIS488/website/OverleafTutorial.pdf}} er en tjeneste som brukes til å lage, redigere og dele akademiske artikler på nettet ved hjelp av LaTeX. Dette er et nettbasert tekstforfattings-program. 
Overleaf v2\footnote{\url{https://no.overleaf.com/learn/how-to/Working_Offline_in_Overleaf}} tilbyr synkronisering med GitHub og Dropbox.

\subsubsection{Ottomatik.io}
Ottomatik\footnote{\url{https://ottomatik.io/}} er en webbasert tjeneste for automatisk sikkerhetskopiering av filer og MySQL databaser.

\subsubsection{Amazon S3}
Amazon Simple Storage Service\footnote{\url{https://aws.amazon.com/s3/}} er en webbasert tjeneste som tilbyr objekt-lagring. Tjenesten tilbyr meget god skalerbarhet, datatilgjengelighet, sikkerhet og ytelse.

KILDE: https://dash.harvard.edu/handle/1/24829568

\subsubsection{HTTP/2}
HTTP/2\footnote{\url{https://www.rfc-editor.org/rfc/pdfrfc/rfc7540.txt.pdf}} er en revisjon av HTTP-nettverksprotokollen. HTTP er et sett med regler for overføring av filer (tekst, grafiske bilder, lyd, video og andre mediefiler). Et av de store målene med HTTP/2 var å tillate multipleksing.

\subsubsection{HTTPS}
HTTPS\footnote{\url{https://www.rfc-editor.org/rfc/pdfrfc/rfc2616.txt.pdf}} (Hypertext Transfer Protocol Secure) er en sikker versjon av HTTP-protokollen som overfører data på nett. HTTPS-protokollen krever at det opprettes en kryptert kanal mellom nettleseren og webserveren, slik at overføring mellom disse blir sikker.

\subsubsection{TLS}
Transport Layer Security er en kryptografisk protokoll. Dette er viktig for å sikre informasjon som overføres gjennom nettverk. TLS brukes i forskjellige tjenester som HTTPS, FTPS, VoIP og VPN.
Når en tjeneste bruker en sikker tilkobling, legges bokstaven S til tjenestens protokollnavn. For eksempel: HTTPS, SMTPS, FTPS, SHIPS.
SSL (Secure Socket Layer) er en utdatert forgjenger til TLS.

KILDE: Thomas, S. (2000). SSL and TLS essentials. New Yourk, 3.

\subsubsection{Let’s Encrypt}
Let’s Encrypt\footnote{\url{https://letsencrypt.org/}} er en gratis og åpen sertifikatautoritet som gir ut X.509-sertifikater for kryptering av transportlaget. Tjenesten leveres av ISRG (Internet Security Research Group).

\subsubsection{HTML}
Hypertext Markup Language\footnote{\url{https://www.w3.org/standards/webdesign/htmlcss}} er et markeringsspråk som brukes til å lage nettsidedokumenter. Det er et system som identifiserer og beskriver de forskjellige komponentene i et dokument som overskrifter, avsnitt og lister. HTML5 er den nyeste versjonen av HTML-standarden.

\subsubsection{CSS}
Casecading Style Sheet\footnote{https://www.w3.org/standards/webdesign/htmlcss} er et format for stilsett som beskriver utseendet i en applikasjon eller på et nettsted. CSS styrer blant annet fonter, farger, bakgrunnsbilder, linjeavstand og sidelayout. Formatet lar utviklere style innhold på en slik måte at brukere enklere kan identifisere hva det er, samt være i tråd med en bedrift eller person sin identitet.

\subsubsection{SASS}
Syntactically Awesome Style Sheets\footnote{\url{https://sass-lang.com/}} er en utvidelse av CSS. SASS gir mulighet til å bruke variabler, nestede regler, inline-import og mixins. Alt dette gjør det enklere å gjenbruke CSS syntaks. Med SASS er det mulig å lage stilark raskere. SASS er kompatibel med alle versjoner av CSS. Det er ingen nettlesere som kjører SASS, man må derfor kompilere SASS til CSS for å kunne bruke det på nettsider.

KILDE: Cederholm, D. (2013). Sass for web designers. A Book Apart.

\subsubsection{JavaScript}
JavaScript\url{https://developer.mozilla.org/en-US/docs/Web/JavaScript} er et programmeringsspråk som følger ECMAScript-spesifikasjonen. Det er mye brukt til å utvikle webapplikasjoner. Språket støttes av de fleste moderne nettlesere, og brukes til å manipulere elementene på nettsiden og stilene som er brukt på dem. JavaScript støtter objektorientert- og funksjonell programmering.

\subsubsection{React}
React\footnote{\url{https://reactjs.org/}} er et fleksibelt JavaScript-bibliotek for å bygge brukergrensesnitt, og brukes som visningslaget for webapplikasjoner som følger MVC. React har åpen kildekode og er utviklet av Facebook.

Alternativ kilde: \url{https://www.fullstackreact.com/assets/media/sGEMe/MNzue/30-days-of-react-ebook-fullstackio.pdf}

\subsubsection{REST-API}
Et API (programmeringsgrensesnitt) er et sett med funksjoner, prosedyrer, metoder eller klasser som brukes av dataprogrammer for å be om tjenester fra operativsystemet eller programvare som er på datamaskinen. En programmerer kan bruke API-er til å lage applikasjoner.

REST (Representational State Transfer) er en arkitektonisk stil for programvare. Et REST-API er et API som følger REST-stilen og de begrensninger som REST definerer.

KILDE: Masse, M. (2011). REST API Design Rulebook: Designing Consistent RESTful Web Service Interfaces. " O'Reilly Media, Inc.".
side 5 og 6

\subsubsection{NODE.JS ?????}

\subsubsection{Figma}
Figma\footnote{\url{https://www.figma.com/}} er et webbasert designverktøy som åpner muligheten for å samarbeide i sanntid. Med Figma er det også mulig å designe mockups og prototyper av applikasjoner og nettsider.

ekstra kilde: \url{https://www.xfive.co/blog/figma-best-designer-developer-cooperation/}

\subsubsection{Google Analytics}
Google Analytics\footnote{\url{https://analytics.google.com/analytics/web/}} er en gratis, webbasert tjeneste som gir statistikk og grunnleggende verktøy for analyse av bruksdata, søkemotoroptimalisering og markedsføring.

\subsubsection{Google Lighthouse}
Google Lighthouse\footnote{\url{https://developers.google.com/web/tools/lighthouse/}} er et gratis verktøy for å sjekke kvaliteten på nettsider. Det analyserer nettsidens tilgjengelighet, hastighet og SEO, samt om nettsiden følger de beste praksiser både generelt sett og for progressive webapplikasjoner. Lighthouse kan kjøres via Google Chrome DevTools, en nettleserutvidelse i Google Chrome eller via nettsiden \url{https://web.dev/}.

\subsubsection{WAVE}
Web Accessibility Evaluation Tool\footnote{\url{https://wave.webaim.org/about}} er et verktøy som tester universell utforming hos nettsider og hvorvidt de følger visse punkter i WCAG 2.1-standarden og Section 508. Verktøyet er tilgjenglig som en nettleserutvidelse og som en webbasert tjeneste hos \url{https://wave.webaim.org/}.

\section{Testing}

\subsection{Funksjonell testing}
Funker nettstedet slik som Sirkus Media vil?

\subsection{Ikke funksjonell testing}
Funker nettstedet bra, teknisk sett?

\subsection{Brukertesting}
Hva syns brukere om nettstedet?






\section{Akademiske og tekniske dokumenter}

Akademiske og tekniske dokumenter ser noe forskjellige ut avhengig av fagfeltet. Likevel, en litt grundigere analyse avslører en felles hovedstruktur, som egentlig er ganske intuitiv\dots og omtrent lik den vi lærte på barneskolen: {\em Hode, kropp og hale}. Hver av disse hoveddelene består som regel av omtrent de samme elementene. 



\subsection{Mayfield Handbook of Technical \& Scientific Writing}
\label{sec:mayfield}

I følge engelsk terminologi deler man gjerne et dokument opp i {\em Front matter, Body og End (Back) matter}, og det gjøres konsekvent i
Mayfield Handbook of Technical \& Scientific Writing \cite{perelman97mht}. Denne utmerkede kilden finnes også på 
elektronisk form\footnote{\url{http://www.mhhe.com/mayfieldpub/tsw/home.htm}}. 
I kapittelet 
{\em Elements of Technical Documents}\footnote{\url{http://www.mhhe.com/mayfieldpub/tsw/elemtech.htm}}
blir den følgende generiske strukturen foreslått:

\begin{compactitem}
\item Front Matter
\begin{compactitem}
\item Title page
\item Abstract
\item Table of contents
\item List of figures
\item List of tables
\item List of terms
\item Acknowledgments
\end{compactitem}

\item Body
\begin{compactitem}
\item Introduction
\item Background
\item Theory
\item Design criteria
\item Materials and apparatus
\item Procedure
\item Workplan
\item Results
\item Discussion
\item Conclusion
\item Recommendations
\end{compactitem}

\item End Matter
\begin{compactitem}
\item References
\item Appendixes
\item Index
\end{compactitem}

\end{compactitem}

Hvert av punktene blir nærmere beskrevet, med mange eksempler på godt (og dårlig) innhold, og det anbefales på det sterkeste å  studere dette kapitellet nærmere. 

Denne strukturen er så nær man kommer en universell mal for en teknisk/vitenskapelig rapport.

\subsection{HiØ/IT}
\label{sec:hiof-it}

Ved HiØ/IT har man valgt å skille grovt sett mellom to typer dokumentasjon.
{\em Prosessdokumentasjonen} omfatter en forprosjektrapport samt individuelle refleksjonsnotater, og selve produktet er omtalt i
{\em hoveddokumentet}\footnote{\url{https://wiki.hiof.no/index.php/Bacheloroppgaven_-_Leveranser}}(eller rett og slett det vi kaller bacheloroppgaven).
Når det gjelder oppbygging og innhold er hoveddokumentet å regne som en typisk akademisk/teknisk rapport/artikkel:

\begin{compactitem}
\item Tittelside
\item Sammendrag
\item (Takk til)
\item Innholdsfortegnelse
\item (Liste over figurer)
\item (Liste over tabeller)
\item Introduksjon
\item Analyse
\item Design
\item Implementasjon
\item Evaluering
\item Diskusjon
\item Konklusjon
\item Referanser
\item Referanser
\end{compactitem}

\section{Skriveverktøy for store og komplekse dokumenter}

For å starte på et stort og komplekst dokument, vil mange åpne sin vanlige teksteditor og sette igang. Det går sjelden bra. Som ved alle typer større oppgaver, bør man velge sine verktøy med omhu.
Før man setter igang å lage et digitalt dokument, kan det være fornuftig å dvele litt ved hvordan dokumenter har blitt, og blir, produsert på tradisjonelt vis. 

\subsection{Forfatter, redaktør, typograf, trykkeri}

Tradisjonelt hersker det en streng arbeidsdeling innen produksjon av dokumenter, som f.eks. en bok, eller en avis. Forfatteren (eller journalisten) skriver såkalt {\rm brødtekst}, som rett og slett er tekst, skrevet for hånd, på skrivemaskin, eller talt inn på diktafon. I teksten er det gjerne markert ulike strukturelle elementer. Det er markert hvor et nytt avsnitt begynner, kanskje det står skrevet at her kommer det en illustrasjon, etc. Dette råmaterialet går gjerne til en redaktør, som kvalitetssikrer innholdet og tildels struktur, og eventuelt foreslår/foretar endringer. Deretter tar typografen (eller den grafiske formgiveren) over, og bestemmer skriftstørrelser, fonter, marger, linjeavstand etc. Til slutt går trykkoriginalen til trykkeriet\footnote{Her kan jo være på sine plass å tenke litt på XHTML og CSS, og hvor de har sine røtter\dots og da tenker jeg ikke på Håkon Wium Lie \smiley.}.

Denne framgangsmåten ivaretar en klar tredeling av arbeidet; innhold, struktur, og form. Innholdet, og grovstrukturen, bestemmes av forfatteren. Redaktøren sørger for å kvalitetssikre innholdet, og eventuelle justere strukturen. Den endelig formgivingen av dokumentet blir typografens rolle. Det er viktig å merke seg at alle disse tre rollene krever høy kompetanse innen helt forskjellige fagfelt. De fleste mener også at det er ganske unaturlig for en forfatter å legge ned mye tankearbeid i hvilken font og størrelse det bør være på overskriftene. På den annen side ville vel det ikke være rett at typografen blandet seg bort i innholdet.

\subsection{The curse of the cursor \\  \normalsize -eller- \\DTP: DeskTop Publishing eller Dom Tokigas Paradis? }

Og etterhvert kom datamaskinene, som tilsynelatende kunne gjøre ``alt'', og dermed begynte mange å gjøre ``alt'', selv. Teksteditorer og desktop publishing verktøy gjorde det mulig for at en person kunne stå for hele prosessen, både skrive, redigere, og formgi, uten å ha særlig kompetanse, ferdighet eller erfaring i noen av feltene.

Personlig husker jeg godt mitt første møte med DTP, tidlig i IT-utdannelsen. Oppgaven var å lage en liten avis, med bilder og det hele, ved hjelp av en søt liten Mac, som på et mysteriøst vis var knyttet sammen med en skriver som til og med skrev ut illustrasjoner og fotografier i grov oppløsning. Den blinkende markøren i teksteditoren var hypnotisk, halvt innbydende, halvt skremmende. Og så måtte man bare kaste seg ut i det. Skrive noen setninger, markere en del som ``bold'' og fontstørrelse 18, fin overskrift, et par ``enter'', forsette med vanlig font, nei, kanskje en litt mer fancy kalligrafifont, kanskje til og med blå? Og dermed var det gjort, forvirringen var total. Noe senere erfarte jeg at i enkelte svenske miljøer mente man at DTP sto for ``Dom Tokigas Paradis". Innen visse segmenter av reklamebransjen mener jeg fremdeles å se spor etter slike traumatiske møter med datastøttet publisering.

Man kan jo spørre seg om det egentlig er noen forskjell mellom et blankt ark, enten på et skrivebord eller i en skrivemaskin, og en blinkende markør øverst i venstre hjørne i en editor? Uansett svar, det blanke arket gir deg ikke anledning til å eksperimentere med fonter, marger, skriftstørrelse og plassering av bilder. Arket handler kun om innhold, og hvordan det skal struktureres, hvilken passasje som følger en annen, og hvor det begynner et nytt avsnitt.

\subsection{Programmering av dokumenter}

Et par år senere skal jeg produsere min første akademiske artikkel, i \LaTeX~med  AMS stilen. Hæh?

Det tok et par dager før jeg skjønte poenget. Skrive innholdet i filer (ark) i en editor uten formateringsmulighet, sette inn kryptiske kommandoer for å markere avsnitt eller matematiske uttrykk, sette sammen delene i en styrefil, legge til noen kommandoer som avgjorde hvordan dokumentet ble seendes ut. Deretter {\em kompilere} hele greia! Som så resulterte i  en DVI-fil\footnote{DVI: DeVice Independent}, som igjen kunne konverteres til alle mulige trykk- og printbare formater. Og det beste av alt, resultatet, rent formmessig, ble helt strøkent, akkurat som i lærebøkene og artikkelsamlingene.

Dette var jo akkurat som å programmere. Full kontroll.


\section{Digital dokumentproduksjon}

Vi foreslår et sett med minimumskrav til digitale dokumentverktøy, basert på  den forutgående analysen. Deretter gir vi kort beskrivelse av 
\LaTeX~ og OpenOffice Writer, som etter en kort vurdering er de to verktøyene som tilfredstiller de fleste av kravene.

\subsection{Generelle krav}
\label{sec:generelle-krav}

Kravene er utarbeidet delvis med basis i den forutgående analysen, men i hovedsak ut fra forfatterens erfaringer.

\vspace{1em}
\begin{compactdesc}
\item [Plattformuavhengighet] I et prosjekt med flere deltagere, er det direkte arrogant å forvente at alle skal bruke samme plattform, fordi valget av dokumentasjonsverktøy krever dette.
\item [Robusthet] En hovedrapport er ofte relativt stor, og tildels med komplekst innhold (figurer, tabeller etc). Det er ikke uvanlig at selv kostbare og vidt utbredte verktøy kræsjer i en slik kontekst, eller verre, begynner å oppføre seg inkonsistent og uforutsigbart Det å åpne et dokument, bare for å oppleve at formatering som ble foretatt kvelden før er ødelagt, er ingen særlig givende erfaring, særlig ikke hvis det er to timer til rapporten skal leveres inn.
\item [Maler/stiler] Det må kunne være mulig å skifte mal på et og samme dokument uten å gjøre ekstra tilpasninger. Dermed oppnår man et klart skille mellom innhold og utforming. Hvis f.eks. retningslinjene for hovedrapporten endrer seg, så er det bare å lage en ny mal og bruke denne.
\item [Oppsplitting]Det bør kunne være mulig å splitte dokumentet i mindre deler, f.eks. en fil for hvert kapittel. Delene må kunne knyttes sammen i et masterdokument. På denne måten vil det være lett å endre rekkefølgen på komponentene. Da er også strukturen uavhengig av det faktiske innholdet. Oppsplitting er også en forutsetning for å kunne jobbe distribuert, der de ulike deltagerne jobber parallelt på ulike deler av rapporten.
\item [Kryssreferanser]Det må være mulig å bruke dynamiske kryssreferanser, slik at man f.eks. kan referere til en nummerert figur. Hvis da nummeret til figuren endrer seg pga. av at det blir lagt inn en figur tidligere i dokumentet, så må referanse oppdateres automatisk.
\item [Bibliografi]De må være enkelt å referere til en ekstern samling bibliografiske referanser. Det også være enkelt å når som helst endre referansestil (f.eks. IEEE eller Harvard).
\item [Brukermasse]Det bør eksistere en stor og variert brukermasse, noe som i seg selv kan være en antydning om at verktøyet holder mål.
\item [Dokumentasjon]Verktøyet må være godt dokumentert, både den innebyggede dokumentasjonen, samt i ulike fora på nettet.
\item [Versjonskontroll]Det må være enkelt å håndtere versjonering, dvs. at man lett kan rekonstruere tidlige versjoner, og lett se hva som er endret fra versjon til versjon.
\item [Framtids/fortidssikker]Verktøyet bør kunne uten videre behandle dokumenter som er laget i tidligere versjoner. Dette vil også være en indikasjon på at dokumentet du jobber fremdeles ``lever'' i rimelig overskuelig framtid. 
\item [Stave/grammatikkontroll] Minimumskravet er at det er mulig å utføre effektiv stavekontroll, for ulike språk, ikke minst norsk bokmål. I tillegg vil det være et pluss om det er mulig å verifisere grammatikk.
\end{compactdesc}


\subsection{\LaTeX}

Klipt og limt fra Wikipedia\footnote{\url{http://no.wikipedia.org/wiki/LaTeX}}:

\begin{quotation}
\LaTeX~ er et typesettingssystem for dokumentproduksjon. Det skrives normalt LaTeX i dokumenter som ikke har hevet og senket tekst.
\LaTeX~ ble opprinnelig utviklet av Leslie Lamport i 1984 og tilbyr idag en lang rekke dokumentproduksjonspakker som f.eks. automatisk opprettelse av innholdsfortegnelser, lister med tabeller og figurer, inkludering av bilder, kryssreferanser, sideoppsett, bibliografier mm.


LaTeX~ baserer seg på at forfatteren av et dokument kun skal være forfatter av dokumentet og slippe å bry seg om utforming og hvordan ting ser ut; noe han normalt sett ikke er ekspert på uansett. Det forfatteren gjør er å dele opp dokumentet i logiske strukturer før han lar LaTeX~ ta seg av selve oppsettet. LaTeX~ ansees ofte som WYSIWYM (det du ser er det du mener) fremfor WYSIWYG (det du ser er det du får) editor. Dette fører til at det ofte anerkjennes som overlegent fremfor andre skriveprogrammer grunnet at endringer som påvirker hele dokumentet er meget enkle å gjennomføre. De trenger ofte bare en endring et sted, så er endringen gjennomført i hele dokumentet.
\end{quotation}


\subsection{OpenOffice Writer (med kloner)}

Klipt og limt fra Wikipedia\footnote{\url{http://no.wikipedia.org/wiki/OpenOffice}}:

\begin{quotation}
Apache OpenOffice (AOO), er en kontorprogramvare som inneholder tekstbehandlingsprogram, regneark, presentasjonsprogram, tegneprogram og et databaseprogram. AOO er fri programvare, og distribueres av Apache Software Foundation under Apache-lisensen. Det betyr at både programmet og dets kildekode kan lastes ned gratis, og privatpersoner eller bedrifter kan selv gjøre endringer eller reparere feil. 

AOO benytter ODF som sitt primære dokumentformat. Dette er en ISO-standard som blant annet er tatt i bruk av offentlig sektor i Norge og EU for utveksling av digitale dokumenter. I tillegg støttes formatene til Microsoft Office og en rekke andre kontorprogrammer. AOO er multiplattform programvare, tilgjengelig for mange forskjellige operativsystemer, deriblant Microsoft Windows, Mac OS X, Linux, Solaris, FreeBSD, NetBSD og OpenBSD.
\end{quotation}

