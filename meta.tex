\def\category{Webutvikling}
\def\ects{20}
\def\area{Informasjonsteknologi}
\def\free{X}
\def\freeafter{}
\def\freecustomer{}

\def\customer{Sirkus Media AS}
\def\tutor{Einar Krogh}
\def\department{Avdeling for Informasjonsteknologi}
\def\projectnr{BO19-G13}
\def\contact{Hans-Christian Hymer}
\def\abstract{
Å benytte  seg  av  internett  har  i  dagens  samfunn  blitt  et  hverdagslig  gjøremål  for mange. Det har derfor blitt ekstra viktig for bedrifter å være synlig på internett. Formålet med dette prosjektet er å forbedre Sirkus Media sin profil på nett, slik at potensielle kunder får en bedre forståelse av deres produkter og tjenester og dermed tar kontakt gjennom deres nettsted. Rapporten starter med å redegjøre for viktige faktorer når det gjelder nettsteder. Deretter blir det utformet et forslag på et nettsted som følger beste praksis og angir struktur, innhold og design. Videre blir dette forslaget implementert. Det ferdige produktet og resultatet er et nettsted som benytter seg av et headless CMS. I tillegg vil nettstedet være responsivt, universelt utformet, ha god søkemotoroptimalisering og være brukervennlig.


% Det har vært en  økende vektlegging på dokumentasjonen i bacheloroppgavene ved HiØ,  slik at hoveddokumentet nå er grunnlaget for karaktersettingen. Formålet med dette prosjektet er å gjøre det enklere for studentene å produsere dokumentasjon med hensiktsmessig innhold, tradisjonell struktur, og profesjonell utforming. Rapporten starter med å redegjøre for generelle krav til vitenskapelige og tekniske rapporter. Det blir lagt spesielt vekt på kravene som stilles ved HiØ. Det gies en kort oversikt over hvordan man produserer og vedlikeholder dokumenter, både analoge og digitale. Deretter blir det utformet en mal som angir struktur og innhold i hoveddokumentet. Etter en ha utviklet en sett med minimumskrav til programvarene som skal brukes, blir det klart at kun to verktøy er aktuelle: \LaTeX~ og {\em OpenOffice Writer}.  En selvforklarende mal blir implementert i dokumentverktøyet \LaTeX~(en mer eller mindre identisk mal for OpenOffice er beskrevet i prosjektet {\em OpenOffice mal for bacheloroppgaven}). 
}
\def\keyone{Webutvikling}
\def\keytwo{Headless CMS}
\def\keythree{Nettsted}
\author{Bereket Goitom Adhanom, Bjørnar Hagen, Line Sharina Aamodt Hagen}
\title{Utvikling av headless CMS for Sirkus Media}
\date{\today
}

